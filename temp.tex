\def\bbb{\begin{eqnarray}}
\def\eee{\end{eqnarray}}
\def\nnn{\nonumber\\}
\def\dd{{\rm d}}
\def\omu{{\rm k \Omega}}

\documentclass[a4paper,11pt,oneside,openany,dvipdfmx]{jsbook}
%
%
%
\usepackage{amsmath,amssymb}
\usepackage{bm}
\usepackage{graphicx}%,draft
\usepackage{graphicx}
\usepackage{subfigure}
\usepackage{verbatim}
\usepackage{wrapfig}
\usepackage{ascmac}
\usepackage{makeidx}
%\usepackage{listings}
\usepackage{listings,jlisting}
\usepackage{color}
\usepackage{longtable}
\usepackage{lineno}
\usepackage{threeparttable}
\usepackage{color}

\lstset{
language={C},
backgroundcolor={\color[gray]{.85}},
basicstyle={\small},
identifierstyle={\small},
commentstyle={\small\ttfamily \color[rgb]{0,0.5,0}},
keywordstyle={\small\bfseries \color[rgb]{1,0,0}},
ndkeywordstyle={\small},
stringstyle={\small\ttfamily \color[rgb]{0,0,1}},
frame={tb},
breaklines=true,
columns=[l]{fullflexible},
numbers=left,
xrightmargin=0zw,
xleftmargin=3zw,
numberstyle={\scriptsize},
stepnumber=1,
numbersep=1zw,
morecomment=[l]{//}
}

%
\makeindex
%
\setlength{\textwidth}{\fullwidth}
\setlength{\textheight}{40\baselineskip}
\addtolength{\textheight}{\topskip}
\setlength{\voffset}{-0.55in}
%
\newcommand{\diff}{\mathrm{d}}  %微分記号
\newcommand{\divergence}{\mathrm{div}\,}  %ダイバージェンス
\newcommand{\grad}{\mathrm{grad}\,}  %グラディエント
\newcommand{\rot}{\mathrm{rot}\,}  %ローテーション
%
\def\maru#1{{\rm\ooalign{\hfil\lower.168ex\hbox{#1}\hfil \crcr\mathhexbox20D}}}
\makeatletter
\newcommand{\figcaption}[1]{\def\@captype{figure}\caption{#1}}
\newcommand{\tblcaption}[1]{\def\@captype{table}\caption{#1}}
%
\newcommand{\eref}[1]{式(\ref{#1})}
\newcommand{\fref}[1]{図\ref{#1}}
\newcommand{\tref}[1]{表\ref{#1}}
%
\title{HL-LHC ATLASピクセル検出器量産時の品質試験に向けたデータベースシステムの構築}
\author{東京工業大学 理学院物理学系物理学コース 陣内研究室\\奥山広貴(19M00398)}
\date{\today}
\begin{document}
%
%
\maketitle
%
%
\frontmatter
%
\addcontentsline{toc}{chapter}{概要}
\chapter*{Abstract}

The ATLAS is one of the experiments conducted at CERN located at one of the interaction points of the Large Hadron Collider(LHC).
The purposes of the experiment are the precise measurements of the Standard Model (SM) and searches for particles predicted by the theories beyond the SM.

In order to acquire more statistics and to achieve more advanced measurements and searches, 
LHC plans to start a new program called HL-LHC to increase the integrated luminosity.
The target luminosity and integrated luminosity is approximately seven times and ten times higher than those from the on-going LHC project, respectively.
Due to the upgrade, it is required especially for the inner tracking detectors to have more radiation tolerance and high granularity. 
It is planned to replace the ATLAS inner detector to a new detector system called the Inner Tracker(ITk). 
Entire ITk consists of silicon detectors and covers wider solid angle acceptance than the current Inner Detector.
For the production of ITk, we are planning to produce $O(10,000)$ modules and conduct a series of tests for quality control(QC tests) for individual modules. 
Those are carried out repeatedly in the production flow.
All the QC tests should be stored to a central database, which is set up at Unicorn university in Czeck Republic, to record the performance of all the modules. 

``Local database'' system had been developed in previous studies in order to manage data at local production sites and to synchronize information to the central DB. The system has been under test-use among production sites towards full production.
There were several items left to be developed for the module production. 
Particularly those are the funcions specialized for the module production and QC tests and the synchronizing tools between the central database and local database.
For this thesis, new functionalities are developed, for example searching results, synchronizing data between the local and the central database, and also it is validated that we can make use of the whole functionalities of the system, including tools that I developed, using real modules at the laboratory. 

Additionally it is confirmed that we can use the tools with the actual data at the production to measure the processing time of the services.
Concerning a function for searching results, the estimated processing time is $2.6\pm 0.1$[sec] for 84,000 results.
Concerning a function for synchronizing data between databases, 
two options have been implemented, i.e. downloading module information and uploading electrical readout results.
The processing time for the download is $4.0\pm 0.4$[sec] per quad module.
The time for the upload is $1.2\pm 0.1$[min] for 5 readout items of quad modules.

I have succeeded to develop the database functions. 
The system has been largely improved to the level that it can be used for the actual production.
Concerning the necessary functions for managing and analyzing readout results, the developments have been completed.
At the end of this thesis, the list of items to be developed is shown, e.g. synchronizing and analyzing of the other test resutls, supporting tools for the processes of module assembly which will occer in worldwide bases.

\chapter*{概要}

欧州原子力研究機構(CERN)に設置されている大型ハドロン衝突型加速器(LHC)の1つの衝突点にて、ATLAS実験が行われている。
この実験は現在、素粒子物理学の基本理論となっている標準模型の精密測定や標準模型を超えた新粒子の探索を目的とし稼働している。

更なる測定、探索に向けて取得統計数の増加を狙い、LHCでは2025年より加速器をアップグレードし、ルミノシティをあげる計画を予定しており、これをHL-LHCと呼ぶ。
ルミノシティは現在のLHCの約7倍、積分ルミノシティは約10倍となる予定である。

HL-LHCにおいて、検出器には高い放射線耐性や位置分解能の向上など現在のものよりも高い水準が要求される。
そのため、ATLAS実験では最内層に設置している内部飛跡検出器の総入れ替えを予定しており、新しく製造する検出器をInner Tracker (ITk)と呼ぶ。
ITkでは全ての領域でシリコン検出器が搭載され、また現在の内部飛跡検出器よりも広い立体角をカバーする設計となっている。

ITkの製造に向けてピクセルモジュール約10,000台を生産し、各モジュールに対して品質試験を行う予定となっている。品質試験は項目が多く、モジュール組み立て工程の中で何度も行うものである。
モジュール情報及び品質試験の結果は固体性能の保持を目的としてチェコに設置されている中央データベースに保存する必要がある。

また各組み立て機関においてモジュール及び品質試験の情報管理に使用することを目的とした「ローカルデータベースシステム」が先行研究で開発されており、いくつかの機関でシステムの試験運用が行われている。
このシステムは、モジュールの量産及び品質試験での運用に向けては必要不可欠であるいくつかの開発課題が残されていた。
特に品質試験に特化した機能開発や中央データベースとローカルデータベース間の同期機能開発は、量産時には必要となる機能であるが実装されていなかった。

本研究では、モジュール組み立てとその品質試験に向けたデータベースシステム構築を行なった。
ローカルデータベースにおける品質試験管理機能の1つとして結果検索機能やデータベース間の同期機能を開発し、システムの拡張を行なった。
また品質試験におけるデータベース操作の流れを確立し、データベースの機能が一連の流れの中で使用可能であることを確認した。

量産時に想定されるデータを用いて開発機能の処理時間測定を行い、その有用性を評価した。
試験結果検索に関しての処理時間は、84,000の結果数に対して$2.6\pm 0.1$[sec]となった。
データベース間の同期機能にして、モジュールのダウンロード機能と読み出し試験結果のアップロード機能を実装した。
モジュールのダウンロード機能に関して、Quadモジュールのダウンロードに要する処理時間は$4.0\pm 0.4$[sec]であった。
読み出し試験結果アップロードに関して、Quadモジュールの読み出し試験5項目のアップロードに要する処理時間は$1.2\pm 0.1$[min]であった。
この時、結果ページに添付したファイル容量は$3.9$[MB]であった。




\tableofcontents
%

%\listoffigures
%
%\listoftables
%
\mainmatter
\linenumbers
%
\chapter{序論}
欧州原子力研究機構(CERN)に設置されている大型ハドロン衝突型加速器(LHC)では、現在、素粒子物理の基礎となっている標準模型の精密測定や標準模型を超える物理現象の探索が行われている。ATLAS実験はLHC上にある1つの衝突点で行われている実験であり、設置しているATLAS検出器を用いて崩壊粒子の測定が行われている。LHCでは加速器のアップグレード(HL$-$LHC)を予定しており、これに向けてATLAS検出器のアップグレードを行う。この章ではLHC$-$ATLAS実験とそのアップグレード計画について説明する。

\section{LHCについて}
LHCはCERNの地下およそ100 $\rm{m}$に設置されている周長26.7 $\rm{km}$の大型ハドロン衝突型加速器である。
バンチと呼ばれる陽子のかたまりを7 $\rm{TeV}$まで加速し、衝突させる。世界最大エネルギーの加速器である。

陽子ビームの加速は4つの前段加速器を用いて行う。始めに水素ガス中の水素原子から電子を分離することで陽子を生成する。
その後最初の線形加速器(Linear Accelarator: LINAC)で50 $\rm{MeV}$まで加速し、陽子シンクロトロンブースター(Proton Synchrotron Booster: PSB)で1.4 $\rm{GeV}$、陽子シンクロトロン(Proton Synchrotron: PS)で25 $\rm{GeV}$、スーパー陽子シンクロトロン(Super Proton Synchrotron)で450 $\rm{GeV}$まで加速されたのちLHCに入射する。CERNにある加速器の概要を図\ref{LHC_overview}に示す。
LHCには4つの衝突点があり、それぞれALICE(A Large Ion Collider Experiment)、LHCb、CMS(Compact Muon Solenoid)、ATLAS(A
Troidal LHC Apparatus)実験が行われている。それぞれの衝突点には崩壊粒子の飛跡やエネルギーを測定するための検出器が設置されており、それら検出器で取得したデータを元に多様な物理解析が行われている。

\begin{figure}[bpt]\centering
\includegraphics[width=12cm]{LHC_overview}
\caption[LHCの全体像]{LHCの全体像\cite{1-1}}
\label{LHC_overview}
\end{figure}

\section{ATLAS実験}
初めにATLAS実験に用いる座標系と用語について説明する。まず衝突点を原点として定義しており、ビーム軸を$z$軸、これに対して垂直な平面を$x-y$平面とする。
$x$軸方向は原点からみてLHCリングの中心に向かう方向であり、$y$軸は地上に向かう方向である。
方位角$\phi$は$z$軸周りの角度であり、極角$\theta$は$z$軸とのなす角である。大抵、極角$\theta$は以下のようにローレンツ不変量$\eta$で表される。
\bbb
\eta = -\rm{ln tan\left(\frac{\theta}{2}\right)}
\eee


\subsection{ATLAS検出器}
ATLAS検出器の全体図を図\ref{atlas_detector}に示す。
最内装に内部飛跡検出器が設置されていて、次に超電導ソレノイド磁石、カロリメータ、トレノイド磁石、ミューオン検出器の順に設置されている。ビームパイプ以外をほとんど検出器で覆うようなデザインとなっている。

\begin{figure}[bpt]\centering
\includegraphics[width=10cm]{atlas_detector}
\caption[ATLAS検出器]{ATLAS検出器\cite{1-2}}
\label{atlas_detector}
\end{figure}

%\begin{figure}[bpt]\centering
%\includegraphics[width=10cm]{atlas_detector_cross_section}
%\caption[ATLAS検出器]{ATLAS検出器\cite{1-2}}
%\label{atlas_detector_cross_section}
%\end{figure}

\subsection{内部飛跡検出器}
内部飛跡検出器の全体図を図\ref{inner_detector}に示す。
内部飛跡検出器は半径$1.15 \rm{m}$、長さ$7 \rm{m}$の円柱形であり、$|\eta|\leq 2.5$の領域を覆っている。超伝導ソレノイド磁石で2 $\rm{T}$の磁場が$z$方向にかけられる。この検出器はさらに3つの検出器で構成され、内側からピクセル検出器、ストリップ検出器、遷移放射検出器の順に設置されている。

\begin{figure}[bpt]\centering
\includegraphics[width=10cm]{inner_detector}
\caption[内部飛跡検出器]{内部飛跡検出器\cite{1-2}}
\label{inner_detector}
\end{figure}


検出器は$\eta$の範囲によってバレル部とエンドキャップ部に分かれる。図\ref{inner_cross_section}にビーム軸方向の端面図を示す。

\begin{figure}[bpt]\centering
\includegraphics[width=12cm]{inner_cross_section}
\caption[内部飛跡検出器]{内部飛跡検出器\cite{1-4}}
\label{inner_cross_section}
\end{figure}

\subsubsection{ピクセル検出器}

ピクセル検出器の全体図を図\ref{pixel_detector_overview}に示す。
ピクセル検出器はバレル層が4層、エンドキャップ層が6層で構成される。バレル部の最内層はIBL(Insertable B-Layer)と呼ばれ、順にB-Layer、Layer-1、Layer-2となっている。

\begin{figure}[bpt]\centering
\includegraphics[width=10cm]{pixel_detector_overview}
\caption[ピクセル検出器全体]{ピクセル検出器全体\cite{1-5}}
\label{pixel_detector_overview}
\end{figure}

ピクセル検出器は、モジュール構造を持つ検出器の集合である。モジュールを図\ref{pixel_detector}に示す。
このピクセルモジュールの詳細については2章で述べる。
\begin{figure}[bpt]\centering
\includegraphics[width=8cm]{pixel_detector}
\caption[ピクセルモジュール]{ピクセルモジュール\cite{1-2}}
\label{pixel_detector}
\end{figure}


\clearpage
\section{HL-LHC実験アップグレード計画}
上述したように、LHCでは加速器のアップグレードを予定しており、これをHL-LHCアップグレード計画と呼ぶ。詳細を以下に示す。
\subsection{概要}
HL-LHCではルミノシティ呼ばれる陽子バンチ密度を上げることで、衝突確率を大きくし、取得統計数を増やす目的がある。
LHCとHL$-$LHCの比較を表\ref{compare_lhc}に示す。

\begin{table}[tbp]
\begin{center}
\caption[LHCの比較]{LHCの比較\cite{1-6}}
\label{compare_lhc}
  \begin{tabular}{|lll|} \hline
    & LHC & HL$-$LHC \\ \hline
    重心系エネルギー & 14 & 14 \\
    瞬間ルミノシティ[$\rm{cm^{-2}s^{-1}}$] & $1\times 10^{34}$ & $7\times10^{34}$ \\
    積分ルミノシティ[$\rm{fb^{-1}}$] & $300$ & $3,000$ \\
    1衝突あたりのイベント数 & $23$ & $138$ \\ \hline 
  \end{tabular}
\end{center}
\end{table}

LHCの運転計画を表\ref{hllhc_plan}に示す。
2020年8月地点の計画では、2025年の初めよりHL-LHCの導入が始まり、2027年の途中からHL-LHC運転開始の予定となっている。
\begin{figure}[bpt]\centering
\includegraphics[width=12cm]{hllhc_plan}
\caption[HL-LHC運転計画]{HL-LHC運転計画\cite{1-7}}
\label{hllhc_plan}
\end{figure}

\subsection{内部飛跡検出器のアップグレード}
ルミノシティの増加に伴い、検出器には以下のような性能が要求される。
\begin{itemize}
  \item 放射線耐性の向上
  \item 高速読み出し
  \item 検出器の細密化
\end{itemize}

HL$-$LHCに向けてATLAS内部飛跡検出器はアップグレードを予定しており、上記の要求を満たすように日々開発を進めている。
アップグレード後の検出器をITk(Inner Tracker)と呼ぶ。イメージを図\ref{itk_image}に示す。

\begin{figure}[bpt]\centering
\includegraphics[width=10cm]{itk_image}
\caption[ITkのイメージ]{ITkのイメージ\cite{1-3}}
\label{itk_image}
\end{figure}

\subsubsection{ITkの構成と現行との比較}
図\ref{itk_cross_section}にITkのビーム軸方向の断面図を示す。
ITkはピクセル検出器とストリップ検出器で構成される。
ピクセル検出器はバレル、インクラインド、エンドキャップ部で構成され、バレル部は5層となっている。

\begin{figure}[bpt]\centering
\includegraphics[width=10cm]{itk_cross_section}
\includegraphics[width=10cm]{itk_pixel_cross_section}
\caption[ITkの断面図]{ITkの断面図\cite{1-3}}
\label{itk_cross_section}
\end{figure}

ピクセル検出器の配置に関して、現行とITkの比較を表\ref{compare_itk_pixel}に示す。
またモジュール数の比較を表\ref{compare_itk_modules}に示す。

\begin{table}[tbp]
\begin{center}
\caption[ピクセル検出器配置の比較]{ピクセル検出器配置の比較}
\label{compare_itk_pixel}
  \begin{tabular}{|lll|} \hline
    & 現行 & ITk \\ \hline
    $r[\rm{mm}]$ & $33~129$ & $39~279$ \\ 
    $|\eta|$ & $<2.5$ & $<4$ \\ 
    層の数 & 4 & 5 \\ \hline
  \end{tabular}
\end{center}
\end{table}

\begin{table}[tbp]
\begin{center}
\caption[ピクセルモジュール数の比較]{ピクセルモジュール数の比較}
\label{compare_itk_modules}
  \begin{tabular}{|l||ll|ll|ll|} \hline
          & バレル部 &            & インクラインド部 & & エンドキャップ部 & \\ \hline 
    層    & 現行     & ITk        & 現行& ITk          & 現行  & ITk \\ \hline
    1     & $280$    & $192$      & $-$ & $512$        & $-$   & $64$ \\ 
    2     & $286$    & $240$      & $-$ & $520$        & $-$   & $242$ \\ 
    3     & $494$    & $660$      & $-$ & $660$        & $-$   & $320$ \\ 
    4     & $676$    & $960$      & $-$ & $1040$       & $288$ & $352$ \\ 
    5     & $-$      & $1300$     & $-$ & $1300$       & $-$   & $468$ \\ \hline
    合計  & $1736$   & $3352$     & $0$ & $4032$       & $288$ & $1446$ \\ \hline\hline
  \end{tabular}
\end{center}
\end{table}

\subsection{期待される物理}
\subsubsection{ヒッグス粒子の測定}


\chapter{ピクセル検出器}

\section{シリコン検出器}
\subsection{半導体\cite{2-1}}
固体は、絶縁体、半導体、導体の3つに大別できる。物質の電気伝導度に関して、絶縁体は非常に低い値、導体は高い値を持つ。
半導体の電気伝導度はこれらの中間であり、温度、光、磁界および微量の不純物に対し非常に敏感である。この特徴のために半導体はエレクトロニクスにおける最も重要な材料の1つになっている。
半導体は元素半導体と化合物半導体に分けられ、多くの物質がその候補となる。
元素半導体の中で代表的なものとして$\rm{Si}$があげられ、ATLASピクセル検出器に使われる半導体は$\rm{Si}$がベースとなっている。
不純物が入っていない、全ての原子が$\rm{Si}$の半導体を真性半導体と呼ぶ。真性半導体中の$\rm{Si}$は4つの$\rm{Si}$と共有結合を構成し、結晶を作る。(図\ref{pure_silicon})

\begin{figure}[bpt]\centering
\includegraphics[width=8cm]{pure_silicon}
\caption[真性半導体中のシリコン]{真性半導体中のシリコン\cite{2-1}}
\label{pure_silicon}
\end{figure}

真性半導体に対し、$\rm{As}$などの最外殻電子を5つもつ原子を不純物としてドープしたものを$\rm{n}$型半導体、$\rm{B}$などの3つのものをドープしたものを$\rm{p}$型半導体と呼ぶ。
それぞれキャリアとして電子、ホールを持つことになり、キャリア移動の特性を組み合わせて様々なデバイスに応用することができる。

\subsection{pn接合}
$\rm{n}$型半導体と$\rm{p}$型半導体を接合し、その接合部を$\rm{pn}$接合と呼ぶ。この接合は各種半導体素子で様々な形で応用されており、ピクセル検出器にも用いられている。

$\rm{pn}$接合の最も重要な特徴は特定の方向にだけ電流が流れやすい整流性である。図\ref{pn_iv}に示すように正電圧をかけると電流は急速に増加する。
逆方向にかけた場合、始めのうちは電流はほとんど流れない。ある臨界電圧に達すると電流は急激に増大する。

\begin{figure}[bpt]\centering
\includegraphics[width=10cm]{pn_iv}
\caption[$\rm{pn}$接合の電流$-$電圧特性]{$\rm{pn}$接合の電流$-$電圧特性\cite{2-1}}
\label{pure_silicon}
\end{figure}

逆方向電圧をかけた場合、図\ref{depletion_field}に示すように$\rm{pn}$接合付近はキャリアが存在しない空乏層領域が形成される。
この時、それぞれの半導体のエネルギー準位に差が生じている状態となっている。
印加電圧$V$と空乏層幅$W$は以下のような関係がある。
\bbb
W \propto \sqrt{V}
\eee

\begin{figure}[bpt]\centering
\includegraphics[width=10cm]{depletion_field}
\caption[空乏層]{空乏層\cite{2-1}}
\label{depletion_field}
\end{figure}

\subsection{検出原理}
荷電粒子が物質中を通過するとき、以下のBethe$-$Blochの公式によってエネルギーを損失する\cite{2-3}。
\bbb
-\left<\frac{\rm{d}E}{\rm{d}x}\right> &=& Kz^2\frac{Z}{A}\frac{1}{\beta^2}\left(\frac{1}{2}\rm{ln}\frac{2m_ec^2\beta^2\gamma^2T_{max}}{I^2}-\beta^2+\cdots\right)\\
\frac{\rm{d}E}{\rm{d}x}&:& 荷電粒子のエネルギー損失量[\rm{eV\cdot g^{-1}\cdot cm^2}] \\\nonumber
K&:& 4\pi N_Ar_e^2m_ec^2 = 0.307075 [\rm{MeVcm^2}] \\\nonumber
z&:& 荷電粒子の電荷量                          \\\nonumber   
Z&:& 物質の原子番号(\rm{Si} 14)                 \\\nonumber
A&:& 物質の原子量(\rm{Si} 28)\\\nonumber
m_ec^2&:& 電子の静止エネルギー(\rm{0.511MeV}) \\\nonumber
\beta&:& 光速を1とした入射粒子の速度 \\\nonumber
\gamma&:&ローレンツ因子 1/\sqrt{1-\beta^2} \\\nonumber
I&:& 励起エネルギーの期待値(シリコン 137\rm{eV}) \\\nonumber
\eee
また$T_max$は質量$M$の入射粒子による1つの電子への最大運動エネルギー移行であり、以下の式で書ける。
\bbb
T_{{\rm max}} = \frac{2m_ec^2\beta^2\gamma^2}{1+2\gamma m_e/M+\left(m_e/M\right)^2}
\eee

荷電粒子が半導体を通過したとき、そのエネルギー損失量に応じて電子・ホール対が生成し、その量を測定することができる。

\section{新型ピクセルモジュール}

新型ピクセルモジュールの構成と、各部品についての説明を以下で述べる。
\subsection{モジュールの構成}
モジュールの構成を図\ref{module_configuration}に示す。
また、モジュールの信号伝達の様子を模式的に表したものを図\ref{module_electric_overview}に示す。

\begin{figure}[bpt]\centering
\includegraphics[width=10cm]{module_configuration}
\caption[ピクセルモジュールの構成]{ピクセルモジュールの構成}
\label{module_configuration}
\end{figure}

\begin{figure}[bpt]\centering
\includegraphics[width=10cm]{module_electric_overview}
\caption[信号伝達]{信号伝達}
\label{module_electric_overview}
\end{figure}

\subsubsection{モジュールの種類}
Quad, tripletなど

\subsubsection{シリコンセンサー}

\subsubsection{読み出しFEチップ}

\begin{figure}[bpt]\centering
\includegraphics[width=10cm]{fechip_rd53a}
\caption[RD53A]{RD53A\cite{2-1}}
\label{fechip_rd53a}
\end{figure}

\begin{figure}[bpt]
  \begin{center}
  \begin{minipage}{0.33\hsize}
    \includegraphics[width=6cm]{diff_fe}
  \end{minipage}
  \begin{minipage}{0.33\hsize}
    \includegraphics[width=6cm]{lin_fe}
  \end{minipage}
  \begin{minipage}{0.33\hsize}
    \includegraphics[width=6cm]{syn_fe}
  \end{minipage}
  \caption[アナログフロントエンド]{アナログフロントエンド\cite{2-1}}
  \label{analog_fe}
  \end{center}
\end{figure}

\begin{figure}[bpt]\centering
\includegraphics[width=10cm]{digital_fe}
\caption[デジタルフロントエンド]{デジタルフロントエンド\cite{2-1}}
\label{digital_fe}
\end{figure}

\subsubsection{プリント基板}

\begin{figure}[bpt]\centering
\includegraphics[width=10cm]{pcb}
\caption[プリント基板]{プリント基板}
\label{pcb}
\end{figure}

\subsubsection{モジュールキャリア}



\chapter{検出器量産と品質試験}

\section{組み立て工程}
現在モジュールの組み立て工程として以下が設定されている。
\begin{enumerate}
  \item プリント基板・ベアモジュール貼り付け
  \begin{itemize}
    \item hoge
  \end{itemize}
  \item ワイヤー配線
  \begin{itemize}
    \item hoge
  \end{itemize}
  \item ワイヤー保護
  \begin{itemize}
    \item hoge
  \end{itemize}
  \item パリレンコーティング
  \begin{itemize}
    \item hoge
  \end{itemize}
  \item 温度サイクル試験
  \begin{itemize}
    \item hoge
  \end{itemize}
  \item 低温耐久試験
  \begin{itemize}
    \item hoge
  \end{itemize}
\end{enumerate}

流れと各組み立て工程のイメージを図\ref{assembly_flow}に示す。
\begin{figure}[bpt]\centering
\includegraphics[width=14cm]{assembly_flow}
\caption[組み立て工程]{組み立て工程}
\label{assembly_flow}
\end{figure}

\section{品質試験}
各組み立て工程に対して、いくつかの品質試験を行う。行う品質試験の代表的なものを以下に示す。

\subsection{外観検査(Visual Inspection)}

\subsection{質量測定(Mass Measurement)}

\subsection{平坦性測定(Metrology)}

\subsection{センサー電流$-$電圧特性確認(Sensor IV)}

\subsection{FEチップ電流$-$電圧特性確認(SLDO VI)}

\subsection{読み出し試験}

\subsubsection{用いるSWとSetup}

\subsubsection{レジスタ書き換え(Chip Configuration)}

\subsubsection{試験項目詳細}
\begin{itemize}
  \item レジスタ読み出し(Register test)
  \item デジタル回路読み出し(Digital scan)
  \item アナログ回路読み出し(Analog scan)
  \item Threshold測定(Threshold scan)
  \item Thresholdグローバルレジスタ調整(Global threshold tuning)
  \item Thresholdピクセルレジスタ調整、再調整、精密調整(Pixel threshold tuning)
  \item ToTグローバルレジスタ調整(ToT tuning)
  \item ノイズ占有率測定(Noise scan)
  \item スタックピクセル測定(Stuck pixel scan)
  \item クロストーク測定(Crosstalk scan)
  \item バンプ接続確認測定(Disconnected bump scan)
  \item 外部トリガーを用いた測定(External trriger)
\end{itemize}
ここから1つ1つの詳細について説明する。
  
\subsubsection{Threshold調整とピクセル解析}\label{sec:pixel_analysis}
以下の流れで読み出しを行う。
\begin{itemize}
  \item デジタル回路読み出し
  \item アナログ回路読み出し
  \item Threshold測定
  \item Thresholdグローバルレジスタ調整
  \item Thresholdピクセルレジスタ調整
  \item ToTグローバルレジスタ調整
  \item Thresholdグローバルレジスタ再調整、精密調整
  \item Threshold測定
  \item スタックピクセル測定
  \item クロストーク測定
\end{itemize}

digital scan
・Explanation
・Output files


モジュール上のピクセルを解析し、各ピクセルが正常かどうかを判断する。
以下の評価基準で解析をし、不良ピクセルには評価基準に応じた評価名が付けられる

\begin{table}[tbp]
\begin{center}
\caption[ピクセル解析の評価基準]{ピクセル解析の評価基準\cite{3-1}}
\label{pixel_analysis_criteria}
  \begin{tabular}{|lll|} \hline
    評価名 & 読み出し項目 & 不良評価基準 \\ \hline
    Digital Dead      & Digital scan           & Occupancy < 1 \\ \hline
    Digital Bad       & Digital scan           & Occupancy < 98 or Occupancy > 102\\\hline 
    Merged Bump       & Analog scan            & Occupancy < 98 or Occupancy > 102 \\ 
                      & Crosstalk scan         & High Crosstalk\\ \hline
    Analog Dead       & Analog scan            & Occupancy < 1 \\ \hline
    Analog Bad        & Analog scan            & Occupancy < 98 or Occupancy > 102 \\ \hline
    Tuning Failed     & Threshold scan         & カイ二乗=0 \\ \hline
    Tuning Bad        & Threshold scan         & $\pm5\sigma$\\ 
                      & ToT scan               & 0 or 15 \\ \hline
    High ENC          & Threshold scan         & $\pm3\sigma$ \\ \hline
    Noisy             & Noise scan             & Occupancy > $10^{-6}$\\ \hline
    Disconnected Bump & Disconnected bump scan & 未決定 \\ 
                      & Source scan            & Occupancyが平均値の1$\%$ \\ \hline
    High Crosstalk    & Crosstalk scan         & Occupancy>0 with 25e(sync)\\
                      &                        & Occupancy>0 with 40e(lin and diff)\\ \hline 
  \end{tabular}
\end{center}
\end{table}

\subsubsection{簡易読み出し試験}
簡易読み出し試験では以下の項目を扱う。
\begin{itemize}
  \item レジスタ読み出し
  \item デジタル回路読み出し
  \item アナログ回路読み出し
  \item Threshold読み出し
  \item ToT読み出し
  \item バンプ接続確認読み出し
\end{itemize}

\subsubsection{バンプ接合確認試験(Bump bond quality)}
放射線源を用いてバンプ接合の確認を行う。

\subsection{各組み立て工程における品質試験}

各組み立て工程と品質試験項目を図\ref{stage_test_flow}に示す。
\begin{figure}[bpt]\centering
\includegraphics[width=10cm]{figure}
\caption[組み立て工程と対応する品質試験]{組み立て工程と対応する品質試験}
\label{stage_test_flow}
\end{figure}

\section{検出器量産と品質試験データ管理に対する要求}

読み出し試験については特にデータ管理が大変。

\chapter{モジュール情報及び品質試験結果管理システム}
前章で述べたように、モジュール生産及び品質試験を世界中で行う。
これらの情報はデータベースシステムを用いて管理することが決定していて、現在この開発を行っている。
システムについては、ITkの全情報を保存する中央データベースと、各組み立て機関に設置し、データ管理を行うローカルデータベースである。
本章ではこれらのデータベースについて説明する。また、システム開発の中で私が開発を行った機能について詳細に説明する。

%%%%%%%%%%%%%%%%%%%%%%%%%%%%%%%%%%%%%%%%%%%%%%%%%
%%%%%%%%%%%%%%%%%%%%%%%%%%%%%%%%%%%%%%%%%%%%%%%%%
%%%%%%%%%%%%%%%%%%%%%%%%%%%%%%%%%%%%%%%%%%%%%%%%%
\section{中央データベース}
\subsection{中央データベースの概要}
\subsubsection{概要}
中央データベースは、ITkの製造に関する全ての情報の保存を目的として開発されたデータベースである。
ユニコーン大学が開発、運用を行っていて、チェコにデータベースサーバーが設けられている。
ITkは、ピクセル検出機とストリップ検出機にから構成される。
これらを生産するにあたって、シリコンセンサーやフレキシブル基板といった小さな部品から製造を行い、それらを用いたモジュールの組み立て、複数モジュールを搭載したstaveやringの組み立てを経て検出器が完成する。
また各組み立て段階において、動作確認等を目的とした品質試験を行う。
これらの過程における全ての構成部品の情報、及び品質試験結果を中央データベースに保存する。

\subsubsection{意義}
中央データベースを扱う一番の目的は、ITkに用いるモジュールの選別とその配置の決定に用いることである。
品質試験結果を解析、品質のいいモジュールを10,000台の中から選別、ITkに搭載する。
また、モジュールの配置も品質を考慮して決定する。例として$|\eta|$が小さく、粒子が多く通過する場所には品質のいいモジュールを搭載するといった位置決定がなされる。

なんか物理っぽい議論ができればなあ。

次に中央データベースに保存された情報は、検出器運転時の参考値として扱われる。
モジュールを例にだすと、品質試験で読み出し試験を行った際の最適な設定値を中央データベースに保存するため、実際の運転時に参照することができる。
また運転前の状態における検出器の性能、運転前後での性能比較を行うことができる。

\clearpage
%%%%%%%%%%%%%%%%%%%%%%%%%%%%%%%%%%%%%%%%%%%%%%%%%
%%%%%%%%%%%%%%%%%%%%%%%%%%%%%%%%%%%%%%%%%%%%%%%%%
%%%%%%%%%%%%%%%%%%%%%%%%%%%%%%%%%%%%%%%%%%%%%%%%%
\section{ローカルデータベース}
\subsection{ローカルデータベースの意義と概要}
中央データベースでは、前述したようにモジュールの情報のみならずITkに関わるすべての情報を管理する。データベースの機能としては汎用的に使えるようなものになっている。
モジュールの組み立て及びその品質試験に関しては3章で述べたように工程が複数に渡り、行う品質試験の数も多い。
1つの生産現場で多いところでは数千個のモジュールを作ることになるため、
データ管理が簡単にかつ円滑に進むようになっているのが好ましい。
このような理由から、生産現場での生産性、利便性に特化し、円滑な生産をサポートすることを目的としたデータベースシステム(ローカルデータベース)を開発している。
システムの概要図を図\ref{localdb_overview}に示す。オープンソースのサービスであるMongoDB\cite{4-1}と、
PythonのウェブフレームワークであるFlask\cite{4-3}を使用し、開発を行なっているローカルデータベース用ウェブアプリケーションを併用することで、データ管理や中央データベースとの同期を行うシステムとなっている。

具体的にローカルデータベースは以下のような利点を持つ。

\begin{itemize}
  \item ローカルにデータベースサーバーを立てるためアクセス速度が早く、円滑にデータ管理を行うことができる。
  \item モジュールの組み立て工程を管理し、生産者の適切な処理を助ける。
  \item モジュールに特化したデータ管理、解析を行うことで異常をいち早く検知できる。
  \item 試験者の情報や試験時間など、品質試験結果以外の必要な情報を正確に管理できる。
\end{itemize}

\begin{figure}[bpt]\centering
\includegraphics[width=13cm]{localdb_overview}
\caption[ローカルデータベースシステムの概要]{ローカルデータベースシステムの概要。各組み立て機関でMongoDBとローカルデータベース用ウェブアプリケーションの立ち上げを行い、独自にデータ管理をするシステムとなっている。品質試験者は図のようにいくつかのソフトウェアを用いて試験結果をMongoDBに保存する。保存された結果はウェブアプリケーションによって閲覧することができる。また結果は中央データベースに集める必要があるため、同期ツールを用いて試験結果の共有を行う。}
\label{localdb_overview}
\end{figure}

\clearpage
%%%%%%%%%%%%%%%%%%%%%%%%%%%%%%%%%%%%%%%%%%%%%%%%%
%%%%%%%%%%%%%%%%%%%%%%%%%%%%%%%%%%%%%%%%%%%%%%%%%
%%%%%%%%%%%%%%%%%%%%%%%%%%%%%%%%%%%%%%%%%%%%%%%%%
\subsection{MongoDBと内部構造\cite{4-2}}
MongoDBとはNoSQLに分類されるデータベースである。
MongoDBの構造について簡単に表したものを図\ref{mongodb_schema}に示す。
一般的なSQLDBのようにテーブル形式ではなく、JSON形式で情報を格納する。
情報を保持している一枚のJSONインスタンスを「\textbf{ドキュメント}」と呼び、「\textbf{コレクション}」と呼ばれる枠に複数のドキュメントが格納されている。
各ドキュメントは「\textbf{ID}」と呼ばれるハッシュ値を持っていて、異なるコレクションにおけるドキュメント間の紐付けはこのIDを用いて行う。

ローカルデータベースシステムにおいて、MongoDBを使用する主な利点を以下に示す。

\begin{itemize}
  \item 各コレクションに格納するドキュメントの構造が動的であるため、開発を柔軟に行うことができる。
  \item JSON形式でデータを保持するため情報取得の際の整形処理をが容易であり、ウェブアプリケーションとの親和性が高い。
  \item データのキャッシュをメモリ上に置き処理を実行するため、高速な読み書きが可能。
\end{itemize}

モジュール及び品質試験に用いる主なコレクションと内部情報を表\ref{localdb_structure}に示す。

\clearpage
%%%%%%%%%%%%%%%%%%%%%%%%%%%%%%%%%%%%%%%%%%%%%%%%%
%%%%%%%%%%%%%%%%%%%%%%%%%%%%%%%%%%%%%%%%%%%%%%%%%
%%%%%%%%%%%%%%%%%%%%%%%%%%%%%%%%%%%%%%%%%%%%%%%%%
\begin{figure}[bpt]\centering
\includegraphics[width=12cm]{mongodb_schema}
\caption[MongoDBの構造の例\cite{4-2}]{MongoDBの構造の例\cite{4-2}。図のようにMongoDBではJSON形式でデータを格納する。1枚のJSONインスタンスをドキュメントと呼び、複数のドキュメントが格納されている枠組みをコレクションと呼ぶ。ドキュメントの構造及びコレクション間の関係等を決めることでデータベースの構造を定義する。}
\label{mongodb_schema}
\end{figure}

\begin{table}[btp]
\begin{center}
\caption[品質試験に用いる主なコレクション]{品質試験に用いる主なコレクション。ローカルデータベースシステムにおいて、MongoDB内に2つのデータベースを設置し、使用する。}
\label{localdb_structure}
  \small
  \begin{tabular}{|lll|} \hline
    データベース名 & コレクション名 & 情報 \\ \hline
    localdb      & component & モジュール情報、FEチップ情報 \\ 
                 & childParentRelation & FEチップとモジュールの関係性 \\ 
                 & QC.module.status & 各モジュールに対する組み立て工程及び選択された試験結果 \\ 
                 & QC.result & 品質試験結果 \\ 
                 & testRun & 読み出し試験結果 \\ 
                 & user & 読み出し試験実施者 \\
                 & institute & 読み出し試験実施場所 \\
                 & componentTestRun & componentとtestRunの関係性 \\
                 & comments & コメント情報 \\ \hline
    localdbtools & QC.status & 組み立て工程及び試験項目\\
                 & viewer.user & 登録ユーザの情報 \\
                 & viewer.query & 読み出し結果キーワード、検索機能実行時に使用 \\ 
                 & viewer.tag.docs & モジュールや試験結果に付けるタグの情報 \\ \hline
  \end{tabular}
\end{center}
\end{table}

\clearpage
%%%%%%%%%%%%%%%%%%%%%%%%%%%%%%%%%%%%%%%%%%%%%%%%%
%%%%%%%%%%%%%%%%%%%%%%%%%%%%%%%%%%%%%%%%%%%%%%%%%
%%%%%%%%%%%%%%%%%%%%%%%%%%%%%%%%%%%%%%%%%%%%%%%%%
\subsection{ウェブアプリケーション} \label{sec:web_app}

各組み立て機関において、試験者が品質試験結果を閲覧、管理するツールとして、ウェブアプリケーションを提供している。
アプリケーション開発には、PythonのウェブフレームワークであるFlaskを使用している。
またアプリーケーションにおいてMongoDBとの通信に用いるAPIとして、PythonライブラリであるPyMongo\cite{4-4}を用いている。
ローカルデータベースとアプリケーション間の処理に特化したイメージを図\ref{webapp_process}に示す。
このようにアプリケーションはデータベースとブラウザー、データベース間のインターフェースとなっている。

試験結果を迅速に分かりやすく見るシステムを作り、円滑な生産の補助や異常結果の早期発見を目的としている。
またデータベースの情報管理のみならず、同期ツールや、後述する試験結果解析ツールなどの外部スクリプトの実行、結果取得等、生産時における多くのデータベース操作はこのアプリケーションを用いて行う。

ウェブアプリケーションでは、現在以下の機能を使用することができる。ある品質試験の結果ページを図\ref{viewer_result}に示す。
\begin{itemize}
  \item 登録モジュール情報及び品質試験結果の閲覧、解析
  \item ローカルデータベースにおけるユーザ管理機能
  \item データベース同期実行機能
\end{itemize}

\begin{figure}[bpt]\centering
\includegraphics[width=16cm]{webapp_process}
\caption[ウェブアプリケーション処理のイメージ]{ウェブアプリケーション処理のイメージ。ウェブアプリケーションではMongoDB通信API(PyMongo)を用いて、データベースのコレクションに検索をかけることで情報を取得する。取得した情報は整形されたのちブラウザに送信、中央データベースとの同期等の処理に用いられる。}
\label{webapp_process}
\end{figure}

\begin{figure}[bpt]\centering
\includegraphics[width=14cm]{viewer_result}
\caption[品質試験結果ページの例]{品質試験結果ページの例。図は品質試験項目であるデジタル回路読み出しの結果を表している。図の上部に試験情報や設定値、下部に結果のグラフが表示されているのを確認できる。}
\label{viewer_result}
\end{figure}


\clearpage
%%%%%%%%%%%%%%%%%%%%%%%%%%%%%%%%%%%%%%%%%%%%%%%%%
%%%%%%%%%%%%%%%%%%%%%%%%%%%%%%%%%%%%%%%%%%%%%%%%%
%%%%%%%%%%%%%%%%%%%%%%%%%%%%%%%%%%%%%%%%%%%%%%%%%
\subsection{先行研究と開発課題}
先行研究で開発された領域と、本研究で取り組んだ開発課題を以下に示す。

\subsubsection{中央データベースの内部構造}
\subsubsection{中央データベースとローカルデータベース間の同期ツールの開発}
\subsubsection{ローカルデータベースにおける品質試験に特化したデータ管理と機能提供}
以下の項目を実装した。
\begin{itemize}
  \item ユーザ管理機能及び各種機能
  \item 品質試験結果の登録と組み立て工程の自動更新
  \item 読み出し試験におけるピクセル解析ツールの開発
  \item 読み出し試験結果の検索機能
\end{itemize}

\subsubsection{量産時におけるデータベース操作の流れの確立}
チュートリアル、demo、documentとか書いたよ。

\clearpage
%%%%%%%%%%%%%%%%%%%%%%%%%%%%%%%%%%%%%%%%%%%%%%%%%%%
%%%%%%%%%%%%%%%%%%%%%%%%%%%%%%%%%%%%%%%%%%%%%%%%%%%
%%%%%%%%%%%%%%%%%%%%%%%%%%%%%%%%%%%%%%%%%%%%%%%%%%%
\section{本研究における開発項目}
以下は本研究で開発した項目である。

%%%%%%%%%%%%%%%%%%%%%%%%%%%%%%%%%%%%%%%%%%%%%%%%%%%
%%%%%%%%%%%%%%%%%%%%%%%%%%%%%%%%%%%%%%%%%%%%%%%%%%%
%%%%%%%%%%%%%%%%%%%%%%%%%%%%%%%%%%%%%%%%%%%%%%%%%%%
\subsection{中央データベースの内部データ構造の実装}

モジュール及びその品質試験に関する情報を中央データベースに情報を保存するために、情報構造の定義、実装を行う必要がある。
以下の項目について、中央データベースが提供しているAPIを用いて内部データ構造の定義を行った。
\begin{enumerate}
  \item モジュールの種類とその構成部品(表\ref{pd_module_structure}).
  \item モジュール組み立て工程と付随する品質試験(表\ref{pd_stage_structure}).
\end{enumerate}

また項目1に関して、Quadモジュールに関する例を図\ref{example_module_structure}に示す。

\begin{table}[btp]
\begin{center}
\caption[中央データベースにおけるモジュールの種類と構造一覧]{中央データベースにおけるモジュールの種類と構造一覧。中央データベースにモジュールを登録するときの情報として、モジュールの種類、構成部品を表のように実装した。Triplet、Quadというように、モジュールの種類ごとに登録できるシステムとなっており、PCB(フレキスブル基板)などの対応する構成部品の紐付けも同時に行うことができる。}
\label{pd_module_structure}
  \scriptsize
  \begin{tabular}{|ll|} \hline
    種類 & 構成する部品(数) \\ \hline
    Triplet L0 stave module   &  Single bare module(3) \\
                              &  Triplet stave PCB(1) \\\hline
    Triplet L0 Ring0 module   &  Single bare module(3) \\
                              &  Triplet R0 PCB(1) \\\hline
    Triplet L0 Ring0.5 module &  Single bare module(3) \\
                              &  Triplet R0.5 PCB(1) \\\hline
    L1 quad module            &  Quad bare module(1) \\
                              &  Quad PCB(1) \\\hline
    Outer system quad moudle  &  Quad bare module(1) \\
                              &  Quad PCB(1) \\\hline
    Outer system quad moudle  &  Dual bare module(1) \\
                              &  Dual PCB(1) \\\hline
    Digital triplet L0 stave module   &  Digital single bare module(3) \\
                                      &  Triplet stave PCB(1) \\\hline
    Digital triplet L0 Ring0 module   &  Digital single bare module(3) \\
                                      &  Triplet R0 PCB(1) \\\hline
    Digital triplet L0 Ring0.5 module &  Digital single bare module(3) \\
                                      &  Triplet R0.5 PCB(1) \\\hline
    Digital quad module       &  Digital quad bare module(1) \\
                              &  Quad PCB(1) \\\hline
    Digital L1 quad moudle    &  Digital quad bare module(1) \\
                              &  Quad PCB(1) \\\hline
    Dummy triplet L0 stave module   &  Dummy single bare module(3) \\
                                    &  Triplet stave PCB(1) \\\hline
    Dummy triplet L0 Ring0 module   &  Dummy single bare module(3) \\
                                    &  Triplet R0 PCB(1) \\\hline
    Dummy triplet L0 Ring0.5 module &  Dummy single bare module(3) \\
                                    &  Triplet R0.5 PCB(1) \\\hline
    Dummy quad module       &  Dummy quad bare module(1) \\
                            &  Quad PCB(1) \\\hline
    Dummy L1 quad moudle    &  Dummyl quad bare module(1) \\
                            &  Quad PCB(1) \\ \hline
  \end{tabular}
\end{center}
\end{table}

\begin{figure}[bpt]\centering
\includegraphics[width=6cm]{example_module_structure}
\caption[中央データベース内におけるモジュール構造の一例(Quadモジュール)]{中央データベース内におけるモジュール構造の一例(Quadモジュール)。例としてOuter system quad moduleの中央データベース内の構造を示している。この種類では構成要素としてそれぞれ対応する種類のModule carrier、Bare Module、PCBを持つことがわかる。さらにBare ModuleはFE chipを4、Sensorを1持つことが分かり、Quadモジュールの構造が正しく実装されていることが分かる。}
\label{example_module_structure}
\end{figure}

\begin{table}[btp]
\begin{center}
\caption[中央データベースにおける組み立て工程と付随するテスト項目]{中央データベースにおける組み立て工程と付随するテスト項目。モジュールの組み立て工程及び品質試験を登録するため、表のような構造を実装した。データベース内でこの表に沿った組み立て工程の登録、更新、試験結果のアップロードができるようになった。}
\label{pd_stage_structure}
  \scriptsize
  \begin{tabular}{|ll|} \hline
    組み立て項目 & 付随する組み立て情報及び品質試験項目 \\ \hline
    1. Bare to PCB assembly & Visual Inspection \\ 
                            & Metrology \\
                            & Mass measurement \\
                            & Glue information \\\hline
    2. Wirebonding          & Visual Inspection \\ 
                            & Wirebond information \\
                            & (Wirebond pull test)\\
                            & First power up\\
                            & Sensor IV\\
                            & SLDO VI\\
                            & Chip configuration\\
                            & Pixel failure test\\\hline

    3. Wirebond Protection  & Visual Inspection \\ 
                            & Potting information \\
                            & Sensor IV \\
                            & Register test\\
                            & Readout for basic electrical \\\hline

    4. Parylene Coating     & Visual Inspection \\ 
                            & Palylene information \\
                            & Mass measurement \\
                            & Sensor IV \\
                            & Register test\\
                            & Readout for basic electrical \\
                            & Bump bond quality \\\hline

    5. Thermal Cycling      & Visual Inspection \\ 
                            & Thermal cycling info \\
                            & Sensor IV \\
                            & Register test\\
                            & Readout for basic electrical \\
                            & Bump bond quality \\\hline

    6. Burn-in              & Visual Inspection \\ 
                            & Metrology \\
                            & Mass Measurement \\
                            & First power up\\
                            & Sensor IV\\
                            & SLDO VI\\
                            & Chip configuration\\
                            & Pixel failure test\\\hline

    7. Reception            & \\\hline 
  \end{tabular}
\end{center}
\end{table}


\clearpage
%%%%%%%%%%%%%%%%%%%%%%%%%%%%%%%%%%%%%%%%%%%%%%%%%%%
%%%%%%%%%%%%%%%%%%%%%%%%%%%%%%%%%%%%%%%%%%%%%%%%%%%
%%%%%%%%%%%%%%%%%%%%%%%%%%%%%%%%%%%%%%%%%%%%%%%%%%%

\subsection{データベース同期ツールの開発} \label{sec:interfacing_tool}
モジュールや品質試験の結果のデータ共有のために、中央データベースとローカルデータベースの間で同期が行われる必要がある。
これを行うツールを設計、開発を行った。

ツールの中では中央データベースが開発、提供している中央データベース通信用APIと、節\ref{sec_web_app}で述べたローカルMongoDBと通信するAPIの2つを用いることで情報共有を行っている。

同期ツールのデータ通信のイメージを図\ref{interfacing_tools_system}に示す。
\begin{figure}[bpt]\centering
\includegraphics[width=13cm]{interfacing_tools_system}
\caption[同期ツールのデータ通信のイメージ]{同期ツールのデータ通信のイメージ。本研究で開発を行っているのは図の赤線の領域に対応する同期ツールである。このツールはPythonを用いて開発しており、処理の中でローカルのMongoDBと通信するAPIと、中央データベースが開発、提供をしているAPIを用いることで、2つのデータベース間の同期を行っている。このとき、中央データベースとの通信はhttp通信で行われる。}
\label{interfacing_tools_system}
\end{figure}

特に本研究ではツールの枠組み設計に加えて、以下の機能を実装した。
\begin{itemize}
  \item モジュール及び構成するFEチップ情報のダウンロード機能
  \item 読み出し試験結果のアップロード機能
\end{itemize}

これらの機能のイメージを図\ref{interface_overview}に示す。
実装の詳細及び処理時間測定について8章で述べる。

\begin{figure}[bpt]\centering
\includegraphics[width=11cm]{interface_overview}
\caption[同期機能の概要]{同期機能の概要。本研究ではモジュール情報のダウンロードと読み出し試験結果のアップロード機能を実装した。図に示している情報を同期する機能となっている。}
\label{interface_overview}
\end{figure}

\clearpage
%%%%%%%%%%%%%%%%%%%%%%%%%%%%%%%%%%%%%%%%%%%%%%%%%%%
%%%%%%%%%%%%%%%%%%%%%%%%%%%%%%%%%%%%%%%%%%%%%%%%%%%
%%%%%%%%%%%%%%%%%%%%%%%%%%%%%%%%%%%%%%%%%%%%%%%%%%%

\subsection{ユーザ管理機能及び各種機能}

異常があった際に確認することを目的として、誰が試験を行ったかを記録することが必要である。
また、モジュールの登録や中央データベースとの同期など、データベースの機能使用を制限することも必要である。
これらを目的として、試験者及びデータベース使用者情報の管理システムを開発、実装した。
この詳細について以下に述べる。

\subsubsection{機能概要}
データベース権限の段階として、管理者、権限付きユーザ、一般ユーザの3段階を設けた。
各ユーザが使うことのできる機能を表\ref{user_functions_summary}に示す。

権限付きユーザの機能としてモジュール及び試験結果にコメント、タグをつける機能を実装した。使用したときの様子を図\ref{webapp_comment}、\ref{webapp_tag}に示す。

\subsubsection{ユーザ登録操作}
表\ref{user_functions_summary}において管理者と権限付ユーザの登録について説明する。

データベースシステム導入時に管理者のアカウントを作成する。
コマンドプロンプト上で開発したスクリプトを用いて実行することで管理者登録がなされ、この際ユーザ名とパスワードを入力する。

権限付ユーザについて、全ての品質試験者及びデータベースユーザ機能使用者は管理者によってユーザ登録される必要がある。
登録はウェブアプリケーションを用いて行い、以下の情報を入力する。
\begin{itemize}
  \item ユーザ名 
  \item 氏名
  \item 所属機関
  \item メールアドレス 
\end{itemize}

管理者が登録を完了すると、登録されたメールアドレスに登録完了メールと仮パスワードが届く。
このメールに従い、ウェブアプリケーション上でユーザがパスワード登録を完了する。

このようにメール機能を用いることでパスワード漏洩の防止、管理者操作の削減を目的としている。

\clearpage
%%%%%%%%%%%%%%%%%%%%%%%%%%%%%%%%%%%%%%%%%%%%%%%%%%%
%%%%%%%%%%%%%%%%%%%%%%%%%%%%%%%%%%%%%%%%%%%%%%%%%%%
%%%%%%%%%%%%%%%%%%%%%%%%%%%%%%%%%%%%%%%%%%%%%%%%%%%
\begin{table}[btp]
\begin{center}
\caption[ローカルデータベースユーザ権限及び使用機能一覧]{ローカルデータベースユーザ権限及び使用機能一覧。ローカルデータシステムにおけるユーザとして、管理者、権限付きユーザ、一般ユーザの3つを設けた。全てのユーザがウェブアプリケーションの閲覧をすることができる。管理者、権限付きユーザにはデータベース読み書き権限とウェブアプリケーションログイン権限が与えられ、試験結果のアップロード、アプリケーション上のユーザ機能の実行ができる。また管理者は権限付きユーザを登録することができる。}
\label{user_functions_summary}
  \small
  \begin{tabular}{|lll|} \hline
    ユーザ       & 付加される権限                               & 使用できる機能 \\ \hline
    管理者       & ユーザ管理権限                     & 権限付きユーザ登録機能\\ 
                 & データベース読み書き権限           & \\ 
                 & ウェブアプリケーションログイン権限 & \\ \hline
    権限付ユーザ & データベース読み書き権限           & 試験結果のアップロード\\ 
                 & ウェブアプリケーションログイン権限 & 中央データベースとのデータ同期機能\\ 
                 &                                    & その他ウェブアプリケーションの機能(コメント、タグ)\\ \hline
    一般ユーザ   &                                    & モジュール情報及び試験結果の閲覧 \\ \hline
  \end{tabular}
\end{center}
\end{table}

\begin{figure}[btp]\centering
\includegraphics[width=12cm]{viewer_comment}
\caption[ウェブアプリケーションにおけるコメント機能]{ウェブアプリケーションにおけるコメント機能。権限付きユーザ及び管理者はモジュールや試験結果に対してコメントをすることができる。図のようにページの右側にコメント欄があり、コメントをテキスト形式で記述することができる。}
\label{webapp_comment}
\end{figure}

\begin{figure}[bpt]\centering
\includegraphics[width=12cm]{viewer_tag}
\caption[ウェブアプリケーションにおけるタグ機能]{ウェブアプリケーションにおけるタグ機能。権限付きユーザ及び管理者はモジュールや試験結果に対してタグをつけることができる。図は試験結果の一覧ページであり、図の表において一番右の列がつけられたタグを示しており、図ではanomalyやgoodといったタグが付けられていることが分かる。}
\label{webapp_tag}
\end{figure}

\clearpage
%%%%%%%%%%%%%%%%%%%%%%%%%%%%%%%%%%%%%%%%%%%%%%%%%%%
%%%%%%%%%%%%%%%%%%%%%%%%%%%%%%%%%%%%%%%%%%%%%%%%%%%
%%%%%%%%%%%%%%%%%%%%%%%%%%%%%%%%%%%%%%%%%%%%%%%%%%%

\subsubsection{機能の仕組み}
ユーザ登録の際には内部で以下の2つの処理が行われるように実装した。

\begin{enumerate}
  \item MongoDBアカウントの作成、読み書き権限の付与
  \item ウェブアプリケーションで用いるユーザ情報ドキュメントの作成
\end{enumerate}

1の処理を行う理由は、登録ユーザが試験結果をmongoDBにアップロードできるようにするためである。
2の情報は、ウェブアプリケーション内でのログイン判断、ユーザの情報保持に使う。
この情報は表\ref{localdb_structure}のviewer.userに保存される。
2つの処理について、実際に保存されるドキュメントの例をリスト\ref{user_doc_1}、\ref{user_doc_2}以下に示す。

\begin{lstlisting}[basicstyle=\scriptsize,caption=MongoDBアカウント情報を持つドキュメントの例。リスト中の"roles"より、localdbとlocaldbtoolsの読み書き権限が付加されていることが分かる。,label=user_doc_1]
{
	"_id" : "localdb.hokuyama",
	"userId" : UUID("fee321eb-83b8-434a-a4a0-fff638b5db36"),
	"user" : "hokuyama",
	"db" : "localdb",
	"credentials" : {
    ...
	},
	"roles" : [
		{
			"role" : "readWrite",
			"db" : "localdb"
		},
		{
			"role" : "readWrite",
			"db" : "localdbtools"
		}
	]
}
\end{lstlisting}
\begin{lstlisting}[basicstyle=\scriptsize,caption=ウェブアプリケーションで扱うユーザ情報を持つドキュメントの例。リスト\ref{user_doc_1}で示したものとは別に、ウェブアプリケーション内でユーザ情報を扱うためにこのドキュメントを保持する必要がある。ウェブにおいてログインはこのドキュメントの存在確認をもってなされる。パスワードはhash化して保存している。,label=user_doc_2]
{
	"_id" : ObjectId("5f0bbe84ef87af2628865de7"),
	"sys" : {
		"rev" : 0,
		"cts" : ISODate("2020-07-13T10:53:07.943Z"),
		"mts" : ISODate("2020-07-13T10:53:07.943Z")
	},
	"username" : "hokuyama",
	"name" : "Hiroki Okuyama",
	"auth" : "readWrite",
	"institution" : "Tokyo Institute of Technology",
	"Email" : "okuyama@hep.phys.titech.ac.jp",
	"password" : "5f4dcc3b5aa765d61d8327deb882cf99"
}
\end{lstlisting}

\clearpage
%%%%%%%%%%%%%%%%%%%%%%%%%%%%%%%%%%%%%%%%%%%%%%%%%%%%%%%
%%%%%%%%%%%%%%%%%%%%%%%%%%%%%%%%%%%%%%%%%%%%%%%%%%%
%%%%%%%%%%%%%%%%%%%%%%%%%%%%%%%%%%%%%%%%%%%%%%%%%%%%%%%
\newpage
\subsection{品質試験結果の登録と組み立て工程の自動更新}
ローカルデータベースへアップロードした品質試験結果の中から、本結果として中央データベースへアップロードする結果を選択する機能を開発した。
品質試験は各モジュール、各組み立て工程に対して行うものであるため、結果選択も同様に工程毎に行うことを想定している。
結果選択後、データベースにおける組み立て工程の情報は次のものへ自動的に更新する機能となっている。

\subsubsection{概要}
あるモジュール、組み立て工程に対して結果を選択する様子を図\ref{webapp_sign_off}に示す。組み立て工程も自動更新されていることがわかる。

\begin{figure}[bpt]\centering
\includegraphics[width=10cm]{webapp_sign_off}
\caption[結果選択画面及び組み立て工程表示の例]{結果選択画面及び組み立て工程表示の例。図の上部で組み立て工程が"MODULETOPCB"である。この段階において結果を選択する処理を行うとローカルデータベース内で選択された結果にタグ付けがなされる。図の中部においてその処理を行なっており、この図では"OPTICAL"と"MASS"の結果を選択している。選択した結果は中央データベースと同期される。また結果選択後は組み立て工程が自動的に更新される。図の下部では"MODULEWIREBONDING"になっていることが分かる。}
\label{webapp_sign_off}
\end{figure}

\newpage
\subsubsection{仕組み}
リスト\ref{qc_status}、\ref{qc_module_status}のようなドキュメントを作成、保存する。
リスト\ref{qc_status}は全てのモジュールに対して共通のドキュメントであり、組み立て工程と各工程における品質試験項目を記録する。
これらの情報は中央データベースより取得される。
この情報を参照することでローカルデータベース内部での組み立て工程の管理が可能となっている。

リスト\ref{qc_module_status}は各モジュールに対して1つ存在し、以下のような情報を保持する。
\begin{itemize}
  \item 現在工程
  \item 各工程における品質試験結果のID
\end{itemize}


\begin{lstlisting}[basicstyle=\scriptsize,caption=組み立て工程及び品質試験一覧情報ドキュメント。このようなドキュメントを作成、保持しておくことで組み立て工程及び品質試験の情報を扱う。ローカルデータベース内に1つこのドキュメントを保持し、品質試験結果選択、組み立て工程の更新時にこのドキュメントを参照する。このドキュメントは中央データベースよりデータ取得して作成する。,label=qc_status]
{
	"_id" : ObjectId("5fc89aa232d56b29091fd64d"),
	"sys" : {
		"mts" : ISODate("2020-12-03T07:58:26.310Z"),
		"cts" : ISODate("2020-12-03T07:58:26.310Z"),
		"rev" : 0
	},
	"dbVersion" : 1.01,
	"proddbVersion" : 1.01,
	"stage_flow" : [
		"MODULETOPCB",
		"MODULEWIREBONDING",
		"MODULEWIREBONDPROTECTION",
		"MODULEPARYLENECOATING",
		"MODULETHERMALCYCLING",
		"MODULEBURNIN",
		"MODULERECEPTION"
	],
	"stage_test" : {
		"MODULETOPCB" : [
			"OPTICAL",
			"GLUE_MODULE_FLEX_ATTACH",
			"MASS",
			"METROLOGY"
		],
		"MODULEWIREBONDING" : [
			"WIREBONDING",
			"OPTICAL",
			"SENSOR_IV",
			"PIXEL_FAILURE_TEST",
			"SLDO_VI",
			"WIREBOND",
			"CHIP_CONFIGURATION"
		],
		"MODULEWIREBONDPROTECTION" : [
			"OPTICAL",
			"POTTING",
			"MASS",
			"READOUT_IN_BASIC_ELECTRICAL_TEST",
			"SENSOR_IV",
			"REGISTER_TEST"
		],
    ...
	},
...
}
\end{lstlisting}

\begin{lstlisting}[basicstyle=\scriptsize,caption=モジュールの組み立て工程及び品質試験結果管理のためのドキュメント例。各モジュールにおいて現在の組み立て工程及び選択された品質試験結果がこのドキュメントに保存される。ドキュメント内の"currentStage"に現工程を保持する。また選択した試験結果のIDを"QC$\_$results"に各組み立て工程ごとに持つようになっている。,label=qc_module_status]
{
	"_id" : ObjectId("5fc4be4c12a45922a91b0e75"),
	"sys" : {
		"mts" : ISODate("2020-11-30T09:41:32.411Z"),
		"cts" : ISODate("2020-11-30T09:41:32.411Z"),
		"rev" : 0
	},
	"dbVersion" : 1.01,
	"proddbVersion" : 1.01,
	"component" : "5fa79114e615fa000a1a5976",
	"currentStage" : "MODULEWIREBONDPROTECTION",
	"latestSyncedStage" : "MODULEWIREBONDING",
	"status" : "created",
	"rework_stage" : [ ],
	"QC_results" : {
		"MODULETOPCB" : {
			"OPTICAL" : "5fc4c2cfb6c93d451e2c9ac1",
			"GLUE_MODULE_FLEX_ATTACH" : "-1",
			"MASS" : "5fc4c2da27766dc6e89c024f",
			"METROLOGY" : "5fc4c2eaf1f19d9cb5859f00"
		},
		"MODULEWIREBONDING" : {
			"WIREBONDING" : "-1",
			"OPTICAL" : "5fc4c4c8b7d0c86912b4958f",
			"SENSOR_IV" : "5fc4c59e9e283a57ccaa1088",
			"PIXEL_FAILURE_TEST" : "5fca342f6e9f1f5eafedfb92",
			"SLDO_VI" : "-1",
			"WIREBOND" : "-1",
			"CHIP_CONFIGURATION" : "-1"
		},
		"MODULEWIREBONDPROTECTION" : {
			"OPTICAL" : "-1",
			"POTTING" : "-1",
			"MASS" : "-1",
			"READOUT_IN_BASIC_ELECTRICAL_TEST" : "-1",
			"SENSOR_IV" : "-1",
			"REGISTER_TEST" : "-1"
		},
    ...
	}
}
\end{lstlisting}

\clearpage
%%%%%%%%%%%%%%%%%%%%%%%%%%%%%%%%%%%%%%%%%%%%%%%%%%%%%%%
%%%%%%%%%%%%%%%%%%%%%%%%%%%%%%%%%%%%%%%%%%%%%%%%%%%%%%%

\newpage
\subsection{読み出し試験結果におけるピクセル解析ツール}
節\ref{sec:pixel_analysis}で述べたように、読み出し試験ではピクセル解析を行う。
これを円滑に行うために、ピクセル解析ツールを開発した。また開発した解析ツールをローカルデータベースシステムに組み込んだ。
このツールについての詳細を以下に示す。

\subsubsection{概要}
YARRで読み出し試験を行った場合、結果ファイル及びディレクトリは各試験項目ごとにわかれて生成される。
また各結果ファイルにはモジュール上の全ピクセル結果がJSONの形で保存されている。

一方、ピクセル解析において、いくつかの試験結果を統一的に扱い、各ピクセルごとに解析を行う必要がある。
そこで、開発した解析ツールでは複数の結果ファイルを1つに統合し、ピクセルごとの解析処理を単純化する役割を担っている。
開発にはPythonとC++を用いた。またCERNが提供している解析フレームワークであるROOT\cite{4-5}を使用し、いくつかの試験データの統一ファイルとして、ROOT内部機能であるTreeを使用した。
このファイル統合処理のイメージを図\ref{analysis_tool_motivation}に示す。

\begin{figure}[bpt]\centering
\includegraphics[width=12cm]{analysis_tool_motivation}
\caption[ピクセル解析ツールにおけるファイル統合処理のイメージ]{ピクセル解析ツールにおけるファイル統合処理のイメージ。YARRの出力ファイル及びディレクトリは\texttt{std$\_$digitalscan}や\texttt{std$\_$thresholdscan}というように読み出し項目ごとである。ピクセル解析ツールでは、図のようにあるモジュールに関連する結果ファイルを統合し、ピクセルごとに行う解析処理を簡易化する狙いがある。}
\label{analysis_tool_motivation}
\end{figure}

実際に作ったTreeファイルと、データ構造のイメージを図\ref{analysis_tool_tree}に示す。

\begin{figure}[bpt]
  \begin{minipage}{0.4\hsize}
    \begin{center}
    \includegraphics[width=6cm]{analysis_tool_tree_file}
    \end{center}
  \end{minipage}
  \begin{minipage}{0.4\hsize}
    \begin{center}
    \includegraphics[width=8cm]{analysis_tool_tree_image}
    \end{center}
  \end{minipage}
  \caption[Treeファイルとそのデータ保持]{Treeファイルとそのデータ保持。実際にこのツールを用いて作ったTreeファイルの内部構造の様子(左)とそのデータ保持のイメージ(右図)を示す。Treeファイルでは、右図のように1つの表に試験結果をまとめている。各行が\texttt{std$\_$digitalscan}といった各読み出し項目に対応し、各列が1ピクセルに対応する。モジュール上の行列(Row、Col)の番号を表の上部に持っておくことで、モジュール上におけるピクセルの位置情報を保持する。}
  \label{analysis_tool_tree}
\end{figure}

\subsubsection{ツールの内部構造と処理の流れ}
開発したツールは、主に以下で説明する3つの実行ファイルで構成される。それぞれの役割について説明する。

\begin{description}
  \item[getData.py (Python)]\mbox{}\\ 
    データベースから対象となるデータファイルを取得、キャッシュファイルとしてサーバー上の一時ディレクトリに保存.
  \item[makeTree (C++)]\mbox{}\\ 
    getData.pyを用いて生成されたキャッシュファイルを読み込み、Treeファイルを作成.
  \item[analysis (C++)]\mbox{}\\ 
  作成したTreeファイルを読み込みピクセル解析を実行、結果値やプロットを出力.
\end{description}

処理の流れのイメージを図\ref{analysis_tool_flow}に示す。
データベースとの通信に関してはMongoDBや現システムとの親和性を考慮し、Pythonを使用した。
Treeファイル作成やその後の解析処理のスクリプトは、ROOTを使用する観点からC++を使用した。
またピクセル解析以外の解析に対しても適応可能とするため、Tree作成部と解析処理部のファイルは分割した。

\begin{figure}[bpt]\centering
\includegraphics[width=10cm]{analysis_tool_flow}
\caption[ピクセル解析ツールの処理の流れ]{ピクセル解析ツールの処理の流れ。ピクセル解析ツールは、図のように3つの実行ファイルにより構成される。getData.py(Python)にてデータベースから結果の取得を行い、makeTree(C++)にてTreeファイルの作成、analysis(C++)にてピクセル解析及び結果のプロットが出力される。}
\label{analysis_tool_flow}
\end{figure}

\clearpage
%%%%%%%%%%%%%%%%%%%%%%%%%%%%%%%%%%%%%%%%%%%%%%%%%%%%%%%
%%%%%%%%%%%%%%%%%%%%%%%%%%%%%%%%%%%%%%%%%%%%%%%%%%%%%%%
%%%%%%%%%%%%%%%%%%%%%%%%%%%%%%%%%%%%%%%%%%%%%%%%%%%
\subsection{読み出し試験結果の検索機能}
登録モジュールや品質試験結果の一覧ページに検索機能を実装した。
確認したいモジュール情報や試験結果を迅速に取得し、閲覧できることを目的としている。検索機能を使用している様子を図\ref{webapp_search_function}に示す。

キーワードを入力し、検索することができる仕組みとなっていて、一般的なウェブページの検索エンジンのように扱うことができる。
現在は単一キーワード検索の他に、以下の機能を実装している。
\begin{itemize}
  \item 完全一致、部分一致検索
  \item AND、OR検索
\end{itemize}

\begin{figure}[bpt]\centering
\includegraphics[width=9cm]{webapp_search_function}
\caption[ウェブアプリケーションにおける検索機能の様子]{ウェブアプリケーションにおける検索機能の様子。図は検索結果一覧表示のページである。図の上部に入力欄があり(赤破線)、ここにキーワードを入力し検索を実行する。図の例では"QU-13"と入力しており、検索結果にはモジュール名QU-13の試験結果が一覧表示されていることが分かる。}
\label{webapp_search_function}
\end{figure}

また生産に向けて、検索にかかる処理時間測定を行った。検索機能の詳しい実装方法と処理時間についての詳細は、6章で述べる。

\clearpage
%%%%%%%%%%%%%%%%%%%%%%%%%%%%%%%%%%%%%%%%%%%%%%%%%%%%%%%
%%%%%%%%%%%%%%%%%%%%%%%%%%%%%%%%%%%%%%%%%%%%%%%%%%%%%%%
%%%%%%%%%%%%%%%%%%%%%%%%%%%%%%%%%%%%%%%%%%%%%%%%%%%%%%%
\subsection{量産時におけるデータベース操作の流れの確立}
量産時におけるデータベース操作の流れを確立した。
以下に従い、モジュール組み立て時におけるデータ管理がなされる。
\begin{enumerate}
  \item 中央データベースへモジュール登録及び登録情報のダウンロード
  \item 1で登録したモジュールに対して品質試験結果のローカルデータベースへのアップロード
  \item ステージ毎に品質試験結果の登録と中央データベースへアップロード
\end{enumerate}

流れのイメージを図\ref{dbsystem_flow}に示す。
品質試験結果のアップロードは各組み立て工程毎に行う。ローカルデータベースで品質試験結果を組み立て工程毎にまとめて扱い、各モジュールの現組み立て工程を正確に管理する目的がある。
\begin{figure}[bpt]\centering
  \begin{minipage}{0.5\hsize}
    \includegraphics[width=7cm]{dbsystem_flowchart}
  \end{minipage}
  \begin{minipage}{0.4\hsize}
    \includegraphics[width=5cm]{dbsystem_flow_image}
  \end{minipage}
\caption[各モジュールにおけるデータベースシステム操作の流れ]{各モジュールにおけるデータベースシステム操作の流れ。モジュール組み立てにおけるデータベース操作の初めに、中央データベースにモジュール登録及びローカルデータベースへモジュール情報のダウンロードを行う(処理1)。その後、ダウンロードしたモジュールに対して組み立て工程に応じた試験結果を生成、ローカルデータベースに保存する(処理2)。各組み立て工程の終わりに試験結果の選択を行い、中央データベースに試験結果を同期する(処理3)。}
\label{dbsystem_flow}
\end{figure}

全組み立て工程が終了すると、モジュールの情報及び品質試験結果が全て中央データベースへ同期されている状態となる。

データベース操作の流れにおいて、開発項目を含め各機能が正常に動くのかの確認として、デモンストレーションを行った。
詳細を5章で述べる。


\chapter{品質試験項目:読み出し試験に用いるソフトウェアと学内実験室におけるデモンストレーション}

\section{読み出し試験に用いるソフトウェアの概要}

\section{読み出し試験結果解析ツールの開発}

\section{学内実験室におけるデモンストレーション}
学内実験室で開発しているソフトウェアを用いて読み出し試験を行い、実際の生産時における流れのデモンストレーションを局所的に行なった。
その詳細について以下に示す。

\subsection{デモンストレーションの流れ}

今回のデモンストレーションで確認した機能を以下に示す。
\begin{itemize}
  \item 中央データベースとローカルデータベースのデータ同期機能(モジュールIDのダウンロード、試験結果のアップロード)
  \item 読み出し試験に使う各種機能(設定ファイル生成、温度モニター、試験結果アップロードと閲覧)
  \item 結果選択とピクセル解析機能
\end{itemize}

またデモンストレーションにおける流れの概要を図\ref{demo_flow}に示す。

\begin{figure}[bpt]\centering
\includegraphics[width=1cm]{figure}
\caption[デモンストレーションの流れ]{デモンストレーションの流れ}
\label{demo_flow}
\end{figure}

\subsection{読み出し試験セットアップ}
読み出し試験に用いるハードウェアのセットアップを表\ref{readout_setup_table}、概要を図\ref{readout_setup_overview}、各ハードウェアの写真を\ref{readout_setup_picture}に示す。
各装置の詳細については付録Bに示す。

\begin{table}[tbp]
\begin{center}
\caption[各ハードウェアの性能]{各ハードウェアの性能}
\label{readout_setup_table}
  \begin{tabular}{|ll|} \hline
    1 & 2 \\ \hline
    result 1 & result 2 \\ \hline 
  \end{tabular}
\end{center}
\end{table}

\begin{figure}[bpt]\centering
\includegraphics[width=1cm]{figure}
\caption[ハードウェアセットアップの概要]{ハードウェアセットアップの概要}
\label{readout_setup_overview}
\end{figure}

\begin{figure}[bpt]\centering
\includegraphics[width=1cm]{figure}
\caption[各ハードウェアの写真]{各ハードウェアの写真}
\label{readout_setup_picture}
\end{figure}

\subsection{機能確認}
\subsubsection{モジュールIDのダウンロード}


\subsubsection{読み出し試験}

・設定ファイル作成

・温度モニター

詳細は付録Bに書くよ。

・試験内容

・試験結果アップロードと閲覧

\subsubsection{結果選択とピクセル解析}

\subsubsection{試験結果アップロード}


\chapter{まとめ}

\section{本論文のまとめ}
HL-LHCに向けてATLAS内部飛跡検出器の総入れ替えを予定しており、これに向けてピクセルモジュールを世界で10,000台生産する予定である。
各モジュールに対して品質試験を行い、全てのモジュール及び品質試験の結果は中央データベースに保存する。

本研究では、この生産及び品質試験に向けてデータベースシステムの構築を行った。
各組み立て機関にてデータ管理をするローカルデータベースを確立し、品質試験結果検索や中央データベースとの同期機能など、生産時に必要となる諸ツールの開発を行った。

開発した諸ツールを含め、生産において必要な機能の確認を行った。本番を想定したソフトウェア、ハードウェアのセットアップと各ツールの処理を実行し、機能が使用可能であることを確認した。

主な開発項目の1つ目として品質試験検索機能を述べた。
開発当初はデータベース内部構造により、試験結果数$n$に対して処理時間が$O(n^2)$かかってしまう問題が発生した。
MongoDB内に新しいコレクションを設け検索に必要な情報を予め1つの場所に保持しておくことにより、処理時間の改善に成功した。
実際に処理時間の測定を行い、データ数の増加に対しても検索機能が不都合なく使えることを確認した。
本番を想定した見積もりを行い、84,000件のデータ数に対して$2.6\pm0.1$[sec]で処理が実行できる見込みであり、生産時において十分に有用な機能であることを確認した。

2つ目に中央データベースとローカルデータベースの同期ツールを開発した。
世界的に使われるツールであり、全ての組み立て機関でこのツールをサポートするために中央データベースへの通信処理時間調査をKEK、LBL、CERNのサーバーを用いて行った。
KEKのサーバーを用いた場合に最も時間がかかることを確認し、このサーバーにおいて十分に使うことができる機能開発を達成すれば世界的に問題がないと考えた。
開発した中央データベース同期ツールについてKEKサーバーを用いて処理速度測定を行った。モジュール情報のダウンロード機能に関して、モジュール1つあたり$4.0\pm 0.4$[sec]の処理時間がかかることを確認した。
処理の詳細を調査すると、モジュールや構成部品の情報取得するために行っている中央データベースAPIの使用に時間がかかっていた。
処理時間の改善策をいくつか考案し、それぞれについて見積もりを行った。
読み出し試験結果のアップロード機能に関して、開発当初はモジュール1つに対して結果アップロード処理時間の見積もり値が$7.9 ± 0.1$[min]であった。
処理時間改善に向けて処理の詳細を調査したところ、結果ファイルの添付処理に大きく時間が要していることが分かった。
ファイル添付についての詳細を調べると、添付処理時間がファイル数とファイル容量に依存していた。
このことから結果ファイルをZIPファイルにまとめ、圧縮しアップロードを行うことで処理時間の改善を図った。
ここで圧縮前後のファイル容量は、それぞれ94、3.9[MB/FEチップ]となった。
結果として処理時間の改善に成功し、その見積もり値が$1.2\pm 0.1$[min]となった。
またファイル添付以外の処理に関して、時間削減の余地がないかを検討した。

\section{現状と今後の課題}
\subsection{ソフトウェアリリースとユーザサポート}
本論文で述べたツールの他に、読み出し試験コマンド統括ソフト、品質試験結果アップロード用ソフトなどの開発もチームとして行っている。
全てのソフトウェアを含めて、品質試験のデータ管理を達成するようなアプリケーションスイートを目指している。
2020年12月9日にファーストバージョンのリリースを行い、いくつかの機関で全体のシステム及びソフトウェアが使われている現状である。

またCERNで行ったチュートリアルを経て、世界的に機能普及が進んでいる。
そのためユーザサポートとしてソフトウェア使用のためのドキュメント\cite{7-1}の作成、整備も行っている。
開発者の連絡先やローカルデータベース専用掲示板へのリンクもドキュメントに記している。何か問題が生じた時などに簡単に問い合わせができる仕組みを整えている。

\subsection{開発課題}
本研究では検索機能や同期機能など、読み出し試験を対象とした機能を重点的に開発した。
ローカルデータベース開発は、読み出し試験の結果を管理したいという要求から始まり、現在はそれ以外の品質試験も含め、全ての結果や組み立て工程の管理も目標としている。
今後の開発課題として以下の機能をあげる。
\begin{itemize}
  \item 読み出し試験以外(外観検査、平坦性測定等)の結果同期機能.
  \item 中央データベースからローカルデータベースへ品質試験結果の同期.
  \item 品質試験結果解析とモジュール選別機能. 
  \item 組み立て工程管理を世界的にサポート.
\end{itemize}

最後の項目に関して、モジュールの組み立て工程は各機関ごとに異なるため、全ての現場における工程を調査しそれをサポートするシステムを実装する必要がある。
例えば、日本では「ベアモジュール・フレキシブル基板貼り付け」工程の後に、モジュール冷却のための構造である「セル搭載」を行う予定であり、これは他の組み立て機関とは異なる。
各地域における柔軟な生産を許しているため、組み立て工程は世界的に細かく統一されていない。
データベースシステムでは多様な組み立て工程に対応できるシステムとする必要がある。

このシステム実装において、中央データベースには選択可能な組み立て工程を定義する機能が存在し、これを使用することを考えている。
そのイメージを図\ref{optional_stage}に示す。

\begin{figure}[bpt]\centering
\includegraphics[width=10cm]{./optional_stage.png}
\caption[中央データベースにおける選択可能な組み立て工程のイメージ]{中央データベースにおける選択可能な組み立て工程のイメージ。図に示すように中央データベースでは選択可能な組み立て工程を定義することができる。図ではセル搭載が選択可能となっていて、ベア・基板貼り付け工程の後に、どちらの工程に進むのかを選択できる。この機能を用いて世界的に多様な組み立て工程をサポートすることを考えている。}
\label{optional_stage}
\end{figure}

以下の開発項目をあげる。
\begin{itemize}
  \item 中央データベースにおいて選択可能な工程定義機能を使い、全ての組み立て機関における工程をサポートする構造を設計、実装.
  \item 本研究で開発した同期ツールを拡張、ローカルデータベース上にも中央データベースと同様の組み立て工程構造を保持.
\end{itemize}

これらを達成することにより、全ての機関における組み立て工程情報の管理ができるようになると考えている。



\chapter{各データベース機能の性能評価}

開発した機能が生産時に十分であるかどうか見積もりは今後の開発を効率よく進めていく上で重要である。
生産時のデータの数や量を想定してその際のシステム性能を見積もることで、今の実装で十分かどうか、改善が必要な場合どのように改善すればいいかを知ることができる。
以下の機能に関して処理時間評価を行った。

\begin{itemize}
  \item ローカルデータベースにおける読み出し試験結果の検索機能
  \item 中央データベースとローカルデータベースのデータ同期機能
\end{itemize}

それぞれの詳細について以下で説明する。

\section{ローカルデータベースにおける読み出し試験結果検索システムの性能評価}
工夫点等。秋の学会で扱った内容。

\section{中央データベースとローカルデータベースのデータ同期機能に関する調査}
ローカルデータベースと中央データベースのデータ同期ツール処理時間に関しての測定を行った。詳細を以下に示す。
\subsection{データ同期ツールに使用するAPI}


\subsection{サーバーの設置場所による処理時間の違い}
4章で述べたように、中央データベースはチェコに設置されている。
そのため試験結果のアップロードに関して、各組み立て機関から接続しデータ送信する処理時間は、機関の場所に大きく依存すると考えられる。
世界的にデータ同期ツールが不自由なく動くことに向けた開発、改善に役立てることを目的として、データを送信する処理時間を、以下の3つの場所に置かれているサーバーを用いて測定した。

\begin{itemize}
  \item 日本、高エネルギー加速器研究所(KEK) 
  \item アメリカ、バークレー研究所(LBL)
  \item スイス、欧州原子核研究機構(CERN)
\end{itemize}

各サーバーの性能を表\ref{server_spec}に示す。また各サーバーが置かれている場所の位置関係を図?に示す。

\begin{table}[tbp]
\caption[サーバーの性能一覧]{サーバーの性能一覧}
\label{server_spec}
\scalebox{0.9}{
  \begin{tabular}{|l|llll|l|l|} \hline
    設置機関 & CPU & & & & Memory & Disk \\
     & Type & Core & Thread & Clock speed[GHz]& [kB] & [GB] \\ \hline 
    KEK & Intel(R) Core(TM) i7-6700 & 4 & 8 & 3.4 & 15,981,000 & 197\\
    LBL & Intel(R) Core(TM) i7-8700 & 6 & 12 & 3.7 & 32,628,000 & 233\\
    CERN & Intel Core Processor (Broadwell, IBRS) & 1 & 10 & 2.2 & 29,978,888 & 80\\ \hline
  \end{tabular}
}
\end{table}

これらのサーバーは実際に生産の際に使用するものと同程度の性能を持ち、サーバーが置かれている環境も生産時と同じであるとしている。
回線の混雑具合などによる処理時間の低下は、本測定では考慮に入れていない。

\subsubsection{登録モジュール情報の取得にかかる時間}
ある登録されたモジュール情報の取得にかかる時間を測定した。
各サーバーでの処理時間についてまとめたものを表\ref{getting_1module}に示す。

\begin{table}[tbp]
\begin{center}
\caption[モジュール情報の取得にかかる時間]{モジュール情報の取得にかかる時間}
\label{getting_1module}
  \begin{tabular}{|ll|} \hline
    サーバー & 処理時間 \\ \hline
    KEK & 0.49 $\pm$ 0.02 \\ 
    LBL & 0.37 $\pm$ 0.02 \\ 
    CERN & 0.30 $\pm$ 0.04 \\ \hline 
  \end{tabular}
\end{center}
\end{table}

\subsubsection{1Byteのデータファイル送信にかかる時間}
ある試験結果ページに、Attachmentとして1Byteのデータファイルを送信する時間を測定した。
各サーバーでの処理時間についてまとめたものを表\ref{sendpd_1byte}に示す。

\begin{table}[tbp]
\begin{center}
\caption[1Byteのデータファイル送信にかかる処理時間]{1Byteのデータファイル送信にかかる処理時間}
\label{sendpd_1byte}
  \begin{tabular}{|ll|} \hline
    サーバー & 処理時間 \\ \hline
    KEK & 0.54 $\pm$ 0.04 \\ 
    LBL & 0.34 $\pm$ 0.03 \\ 
    CERN & 0.39 $\pm$ 0.02 \\ \hline 
  \end{tabular}
\end{center}
\end{table}

\subsubsection{データファイル送信処理時間の容量依存性}
ある試験結果ページに、Attachmentとしてデータファイルを送信する時間をいくつかのファイルサイズにおいて測定した。
ファイルサイズと処理時間の関係を図\ref{datasize_vs_time}に示す。上述した1Byteでの測定点も含んでいる。
\begin{figure}[bpt]\centering
\includegraphics[width=9cm,angle=270]{datasize_vs_time.pdf}
\caption[送信するファイルサイズと処理時間の関係]{送信するファイルサイズと処理時間の関係}
\label{datasize_vs_time}
\end{figure}

\subsection{モジュールIDのダウンロード機能確認と処理時間測定}
\subsubsection{アルゴリズム}
\subsubsection{機能確認}
KEKのモジュール登録とダウンロード機能の確認。
\subsubsection{処理時間測定}
工夫点
\subsection{読み出し試験結果のアップロード機能確認と処理時間測定}
\subsubsection{アルゴリズム}
\subsubsection{機能確認}
\subsubsection{処理時間測定}

工夫点


\chapter{まとめ}

\section{まとめ}

\section{今後の課題}
 
\section{結論}

%
%
\appendix
%
\include{appendix2}
\chapter{RD53Aの回路図とフレキシブル基板} \label{chap:rd53a_circit}

\begin{figure}[bpt]
  \begin{center}
    \includegraphics[width=8cm]{diff_fe}
    \includegraphics[width=8cm]{lin_fe}
    \includegraphics[width=8cm]{syn_fe}
  \caption[アナログフロントエンド]{アナログフロントエンド\cite{2-1}}
  \label{analog_fe}
  \end{center}
\end{figure}

\begin{figure}[bpt]\centering
\includegraphics[width=10cm]{digital_fe}
\caption[デジタルフロントエンド]{デジタルフロントエンド\cite{2-1}}
\label{digital_fe}
\end{figure}

\begin{figure}[bpt]\centering
\includegraphics[width=10cm]{pcb}
\caption[フレキシブル基板]{フレキシブル基板}
\label{pcb}
\end{figure}



\chapter{Appendix A}
付録がいる場合はどうぞ。
\chapter{RD53Aの回路図とフレキシブル基板} \label{chap:rd53a_circit}

\section{アナログ回路}
\begin{figure}[bpt]
  \begin{center}
    \includegraphics[width=8cm]{./diff_fe.png}
    \includegraphics[width=8cm]{./lin_fe.png}
    \includegraphics[width=8cm]{./syn_fe.png}
  \caption[アナログフロントエンド]{アナログフロントエンド\cite{2-1}}
  \label{analog_fe}
  \end{center}
\end{figure}

\begin{figure}[bpt]\centering
\includegraphics[width=10cm]{./digital_fe.png}
\caption[デジタルフロントエンド]{デジタルフロントエンド\cite{2-1}}
\label{digital_fe}
\end{figure}

\section{試験用電荷入射のイメージ}
RD53Aの各ピクセルが持つ試験用電荷入射回路の簡略図を図\ref{injection_circuit}に示す。
\begin{figure}[bpt]\centering
\includegraphics[width=10cm]{./injection_circuit.png}
\caption[RD53Aの各ピクセルが持つ試験用電荷入射回路の簡略図]{RD53Aの各ピクセルが持つ試験用電荷入射回路の簡略図\cite{b-1}。図のように2つの電位を入力し、その差分の電圧を回路内のコンデンサにかける。これを開放することで、電荷をピクセル回路内に入力する。}
\label{injection_circuit}
\end{figure}

\section{フレキシブル基板}
\begin{figure}[bpt]\centering
\includegraphics[width=10cm]{./pcb.png}
\caption[フレキシブル基板]{フレキシブル基板}
\label{pcb}
\end{figure}



\chapter{ファイル送信時におけるデータ容量と処理時間の関係について}

(時間があればもっとちゃんと書きます。)
KEKとLBLにおいてなぜ差が出るのかを考察する。
ファイル送信時におけるデータ容量と処理時間の関係は、線形性を示さない。
図\ref{datasize_vs_time_scp}はKEKからLBLのサーバーにscpコマンドを用いてファイル送信を行い、データ容量と処理時間の関係を取得したものである。
赤線が線形フィットであるが、測定点は優位にずれていることが分かる。
これはTCP通信においてパケットの送信に輻輳制御と呼ばれる技術が使われており、データ送信量を変化させながら情報通信を行っている。

\begin{figure}[bpt]\centering
  \begin{center}
    \includegraphics[width=7cm,angle=270]{datasize_vs_time_scp.pdf}
  \caption[添付するファイルサイズと処理時間の関係]{添付するファイルサイズと処理時間の関係}
  \label{datasize_vs_time_scp}
  \end{center}
\end{figure}

scpによるファイル送信をKEK->LBL、LBL->KEKの場合に対しておこなった。図\ref{datasize_vs_time_kek_lbl}のように差異が見られた。
Serverのspecは同程度。読み書き速度も変わらなかった。
輻輳制御アルゴリズム(Cubic)、Window sizeとping(111msec)は変わらなかった。
一般的には上りより下りの方が太いと考えると、KEKの上りnetworkはLBL上りと比べての方が細いと考えられる。
\begin{figure}[bpt]\centering
  \begin{center}
    \includegraphics[width=7cm,angle=270]{scp_kek_lbl.pdf}
  \caption[KEK、LBL間のファイル送信]{KEK、LBL間のファイル送信}
  \label{datasize_vs_time_kek_lbl}
  \end{center}
\end{figure}

scpファイル送信、KEK-Lxplus、LBL->Lxplus
上述したようにKEKの上りネットワークは細い。
加えてpingによる反応時間がKEKは170msec程度なのに対し、LBLは150msec程度。ネットワーク上の距離差も処理速度影響していると考えられる。
\begin{figure}[bpt]\centering
  \begin{center}
    \includegraphics[width=7cm,angle=270]{scp_to_cern.pdf}
  \caption[KEK、LBLとCERN間のファイル送信]{KEK、LBLとCERN間のファイル送信}
  \label{datasize_vs_time_cern}
  \end{center}
\end{figure}

CERNと中央DBでは条件が違うが、上述したことをまとめるとKEKとLBLの間で処理時間の際が生まれる要因は以下であると考えた。
\begin{itemize}
  \item KEKの上りネットワークが遅い
  \item ネットワーク上の距離差があり、KEKの方が差が大きい。
\end{itemize}


%
%
\begin{thebibliography}{9}

\bibitem[1]{1}
Keysight Technologies. "E3640A – E3649A Programmable DC Power Supplies - Data Sheet". Keysight Technologies. 2018-3-3
https://www.keysight.com/jp/ja/assets/7018-06827/data-sheets/5968-7355.pdf,(参照2020-12-22)

\bibitem[2]{2}
Mouser Electronics. "XpressK7-160-Gen2". Mouser Electronics. 更新日の記載なし
https://www.mouser.jp/ProductDetail/ReFLEX-CES/XpressK7-160-Gen2?qs=rrS6PyfT74eSJLUPLu1P5g\%3D\%3D,(参照2020-12-22)

\bibitem[3]{3}
Microchip Technology. "2.7V Dual Channel 10-Bit A/D Converter with SPITM Serial Interface". 秋月電子通商. 2006-08-12.
https://akizukidenshi.com/download/ds/microchip/mcp3002.pdf,(参照2020-12-22)

\bibitem[4]{4}
RASPBERRY PI FOUNDATION. "Raspberry Pi 3 Model B+". RASPBERRY PI FOUNDATION. 更新日の記載なし.
https://www.raspberrypi.org/products/raspberry-pi-3-model-b-plus/,(参照2020-12-22)



\end{thebibliography}

%
%
\chapter*{謝辞}
\addcontentsline{toc}{chapter}{謝辞}

%
%
\newpage
\printindex
%
%
\end{document}
