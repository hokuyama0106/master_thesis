\chapter{品質試験項目:読み出し試験に用いるソフトウェアと学内実験室におけるデモンストレーション}

\section{読み出し試験に用いるソフトウェアの概要}

\section{読み出し試験結果解析ツールの開発}

\section{学内実験室におけるデモンストレーション}
学内実験室で開発しているソフトウェアを用いて読み出し試験を行い、実際の生産時における流れのデモンストレーションを局所的に行なった。
その詳細について以下に示す。

\subsection{デモンストレーションの流れ}

今回のデモンストレーションで確認した機能を以下に示す。
\begin{itemize}
  \item 中央データベースとローカルデータベースのデータ同期機能(モジュールIDのダウンロード、試験結果のアップロード)
  \item 読み出し試験に使う各種機能(設定ファイル生成、温度、電圧、電流モニタリング、試験結果閲覧)
  \item 結果選択とピクセル解析機能
\end{itemize}

またデモンストレーションにおける流れの概要を図\ref{demo_flow}に示す。

\begin{figure}[bpt]\centering
\includegraphics[width=1cm]{figure}
\caption[デモンストレーションの流れ]{デモンストレーションの流れ}
\label{demo_flow}
\end{figure}

\subsection{読み出し試験セットアップ}
読み出し試験に用いるハードウェアのセットアップを表\ref{readout_setup_table}、概要を図\ref{readout_setup_overview}、各ハードウェアの写真を\ref{readout_setup_picture}に示す。
各装置の詳細については付録Bに示す。

\begin{table}[tbp]
\begin{center}
\caption[各ハードウェアの性能]{各ハードウェアの性能}
\label{readout_setup_table}
  \begin{tabular}{|ll|} \hline
    1 & 2 \\ \hline
    result 1 & result 2 \\ \hline 
  \end{tabular}
\end{center}
\end{table}

\begin{figure}[bpt]\centering
\includegraphics[width=1cm]{figure}
\caption[ハードウェアセットアップの概要]{ハードウェアセットアップの概要}
\label{readout_setup_overview}
\end{figure}

\begin{figure}[bpt]\centering
\includegraphics[width=1cm]{figure}
\caption[各ハードウェアの写真]{各ハードウェアの写真}
\label{readout_setup_picture}
\end{figure}

\subsection{読み出し試験内容}
読み出し試験を通して、モジュールに与える電圧値、電流値、チップ横についているNTCから読み取れる温度を記録した。
以下の流れに沿って読み出しを行なった。

\subsection{機能確認}
\subsubsection{モジュールIDのダウンロード}
登録したモジュールのIDを機能を使ってダウンロードし、ウェブアプリケーションで確認した。
確認した画面を図\ref{download_SCC}に示す。

\begin{figure}[bpt]\centering
\includegraphics[width=1cm]{figure}
\caption[ダウンロードしたモジュールID確認画面]{ダウンロードしたモジュールID確認画面}
\label{download_SCC}
\end{figure}

\subsubsection{読み出し試験}
以下の流れで読み出し試験を行なった。読み出し試験はサーバーのシェルを用いて行う。

・設定ファイル生成

ダウンロードしたモジュールのIDを用いて、読み出しに用いる設定ファイルを生成した。
イメージを図\ref{config_retriever}、実際に生成したファイルを確認した画面を図\ref{create_config}に示す。

\begin{figure}[bpt]\centering
\includegraphics[width=1cm]{figure}
\caption[設定ファイル生成のイメージ]{設定ファイル生成のイメージ}
\label{config_retriever}
\end{figure}

\begin{figure}[bpt]\centering
\includegraphics[width=1cm]{figure}
\caption[生成ファイル確認画面]{生成ファイル確認画面}
\label{create_config}
\end{figure}

・試験実施とアップロード

上述した流れに沿って読み出し試験を実施した。試験結果は各試験の終わりに自動的にアップロードされるようなシステムとなっている。

・電圧値、電流値、温度のモニタリング

記録した値をGrafanaを使ってモニタリングをした。その様子を図\ref{monitoring_dcs}に示す。

\begin{figure}[bpt]\centering
\includegraphics[width=1cm]{figure}
\caption[DCSのモニタリング]{DCSのモニタリング}
\label{monitoring_dcs}
\end{figure}

・試験結果の閲覧

ウェブアプリケーションを用いて、試験結果を閲覧した。その様子を図\ref{view_scan_result}、\ref{view_dcs}に示す。

\begin{figure}[bpt]\centering
\includegraphics[width=1cm]{figure}
\caption[試験結果の閲覧]{試験結果の閲覧}
\label{view_scan_result}
\end{figure}

\begin{figure}[bpt]\centering
\includegraphics[width=1cm]{figure}
\caption[各測定値の閲覧]{各測定値の閲覧}
\label{view_dcs}
\end{figure}

各scanの結果を\ref{scan_result}に示す。

\begin{figure}[bpt]\centering
\includegraphics[width=1cm]{figure}
\caption[各scan結果]{各scan結果}
\label{scan_result}
\end{figure}

\subsubsection{結果選択とピクセル解析}

\subsubsection{試験結果アップロード}

\subsubsection{今後の開発方針}

