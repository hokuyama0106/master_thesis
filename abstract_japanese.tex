\chapter*{概要}

欧州原子力研究機構(CERN)に設置されている大型ハドロン衝突型加速器(LHC)の1つの衝突点にて、LHC-ATLAS実験が行われている。
この実験は現在、素粒子物理学の基本理論となっている標準模型の精密測定や標準模型を超えた新粒子の探索を目的としている。

更なる測定、探索に向けて取得統計数の増加を狙い、LHCでは2025年より加速器をアップグレードし、衝突確率に対応する量である瞬間ルミノシティをあげる計画を予定しており、これをHL-LHCと呼ぶ。
瞬間ルミノシティは現在のLHCの約7倍、積分ルミノシティは約10倍となる予定である。

HL-LHCにおいて、検出器には高い放射線耐性や位置分解能の向上など現在のものよりも高い水準が要求される。
そのため、ATLAS実験では最内層に設置している内部飛跡検出器の総入れ替えを予定しており、新しく製造する検出器をInner Tracker (ITk)と呼ぶ。
ITkでは全ての領域でシリコン検出器が搭載され、また現在の内部飛跡検出器よりも広い立体角をカバーする設計となっている。

ITkの製造に向けてピクセルモジュール約10,000台を生産し、各モジュールに対して品質試験を行う予定となっている。品質試験は項目が多く、モジュール組み立て工程の中で何度も行うものである。
モジュール情報及び品質試験の結果は固体性能の保持やITk運転時における参照を目的としてチェコに設置されている中央データベースに保存する必要がある。

また各組み立て機関においてモジュール及び品質試験の情報管理に使用することを目的とした「ローカルデータベースシステム」が、先行研究で開発されており、いくつかの機関でシステムの試験運用が行われていた。
このシステムはモジュールの量産及び品質試験での使用に向けていくつかの開発課題が残されていた。
特に品質試験に特化した機能開発や中央データベースとローカルデータベース間の同期機能開発は、量産時には必要となる機能であるが実装されていなかった。

本研究では、これらのデータベースシステム構築を行なった。
ローカルデータベースにおける品質試験管理機能の1つとして結果検索機能やデータベース間の同期機能を開発し、システムの拡張を行なった。
また品質試験におけるデータベース操作の流れを確立し、データベース機能全体が流れの中で使用可能であることを確認した。
さらに開発した機能が、量産時に想定されるデータに対して十分に使用可能であるかを、機能の処理時間測定を行い確認した。


