\chapter*{概要}

欧州原子力研究機構(CERN)に設置されている大型ハドロン衝突型加速器(LHC)の1つの衝突点にて、LHC-ATLAS実験が行われている。
この実験は現在、素粒子物理学の基本理論となっている標準模型の精密測定や標準模型を超えた新粒子の探索を目的としている。

更なる測定、探索に向けて取得統計数の増加を狙い、LHCでは2025年より加速器をアップグレードし、衝突確率に対応する量である瞬間ルミノシティをあげる計画を予定しており、これをHL-LHCと呼ぶ。
瞬間ルミノシティは現在のLHCの約7倍、積分ルミノシティは約10倍となる予定である。

HL-LHCにおいて、検出器には高い放射線耐性や位置分解能の向上など現在のものよりも高い水準が要求される。
そのため、ATLAS実験では最内層に設置している内部飛跡検出器の総入れ替えを予定しており、新しく製造する検出器をInner Tracker (ITk)と呼ぶ。
ITkでは全ての領域でシリコン検出器が搭載され、また現在の内部飛跡検出器よりも広い立体角をカバーする設計となっている。

ITkの製造に向けてピクセルモジュール約10,000台を生産し、各モジュールに対して品質試験を行う予定となっている。品質試験は項目が多く、モジュール組み立て工程の中で何度も行うものである。
またモジュール情報及び品質試験の結果は固体性能の保持やITk運転時における参照を目的としてチェコに設置されている中央データベースに保存する必要がある。
私はモジュール量産に向けて、各現場においてモジュール及び品質試験の情報管理と中央データベースとの同期に使用することを目的としたローカルデータベースシステムの開発と機能普及を行った。現在、いくつかの組み立て機関が生産本番に向けて、システムの試験運用を行っている。

データベースシステムの開発に関して、先行研究で開発されたシステムに拡張する形で試験結果検索やデータ同期など量産時に必要となるいくつかのツールを実装した。
また実装した機能を含めデータベース機能全体が使用可能であることを確認した。
さらに開発した機能が量産時のデータ数に対して十分に使用可能であるかを、機能の処理時間測定を行い確認した。


