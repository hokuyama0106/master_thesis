\chapter{まとめ}

\section{本論文のまとめ}
CERNにあるLHC加速器を用いてATLAS実験が行われている。2025年よりLHCのアップグレードを行う予定であり、これをHL-LHCと呼ぶ。
HL-LHCに向けてATLAS内部飛跡検出器の総入れ替えを予定しており、ピクセル検出器は現行のものよりも広い範囲をカバーする。
新しく作る検出器はITkと呼ばれ、これに向けてピクセルモジュールを世界で10,000台生産する予定であり、各モジュールに対して品質試験を行う。
全てのモジュール及び品質試験の結果は中央データベースに保存する。

本研究では、この生産及び品質試験に向けてデータベースシステムの構築を行った。
各生産現場にてデータ保存、管理をするローカルデータベースを確立し、品質試験結果検索や中央データベースとの同期機能など、生産時に必要となる諸ツールの開発を行った。
またデータベース機能の普及と生産の成功に向けて、共同生産者及びシステムユーザを対象としたシステムのチュートリアルを行った。
多くの議論とフィードバックを受け、システムの更なる改善に繋げた。チュートリアルを受けて2020年11月現在、世界18の生産現場でローカルデータベースの導入及び試運転が行われていることを確認し、システム普及を達成した。

開発した諸ツールを含め、生産において必要な機能の確認を学内実験室にて行った。本番を想定したソフトウェア、ハードウェアのセットアップと各ツールの処理実行を達成し、機能が使用可能であることを確認した。

主な開発項目の1つ目として品質試験検索機能を開発し、MongoDB内に新しいコレクションを設ける工夫により、開発当初に問題となった処理時間の改善に成功した。
実際に処理時間の測定を行い、データ数の増加に対しても検索機能が不都合なく使えることを確認した。
本番を想定した見積もりを行い、84,000件のデータ数に対して$2.6\pm0.1$[sec]で処理が実行できる見込みであり、生産時において十分に使える機能であることを確認した。

2つ目に中央データベースとの同期ツールを開発した。
世界的に使われるツールであり、全ての生産現場でこのツールをサポートするために中央データベースへの通信処理時間調査をKEK、LBL、CERNのサーバーを用いて行った。
KEKのサーバーを用いた場合に最も時間がかかることを確認し、このサーバーにおいて十分に使うことができる機能開発を達成すれば世界的に問題がないと考えた。
開発した中央データベース同期ツールについてKEKサーバーを用いて処理速度測定を行った。モジュール情報のダウンロード機能に関して、モジュール1つあたり$4.0\pm 0.4$[sec]の処理時間がかかることを確認した。
生産に向けて処理時間の改善策をいくつか考案し、それぞれについて見積もりを行った。
読み出し試験結果のアップロード機能に関して、処理時間のボトルネックとなっている箇所を分析し、結果ファイルをZIPファイルにまとめ、圧縮しアップロードを行うという改善を加え、処理時間の改善に成功した。
生産時においてモジュール1つに対しての結果アップロード処理時間の見積もりが$1.2\pm 0.1$[sec]であり、生産を通して使える機能であることを確認した。

\section{現状と今後の課題}
\subsection{ソフトウェアリリースとユーザサポート}
ローカルデータベース開発は複数人のチームで行っている。
また、本論文で述べたツールの他に、読み出し試験コマンド統括ソフト、品質試験結果アップロード用ソフトなどの開発もチームとして行っている。
全てのソフトウェアを含めて、品質試験のデータ管理を達成するようなアプリケーションスイートを目指している。
2020年12月9日にファーストバージョンのリリースを行い、いくつかの機関で全体のシステム及びソフトウェアが使われている現状である。

またCERNで行ったチュートリアルを経て、世界的に機能普及が進んでいる。
そのためユーザサポートとしてソフトウェア使用のためのドキュメント[7-1]の作成、整備も行っている。
開発者の連絡先やローカルデータベース専用掲示板へのリンクもドキュメントに記している。何か問題が生じた時などに簡単に問い合わせができる仕組みを整えている。

\subsection{開発課題}
本研究では検索機能や同期機能など、読み出し試験を対象とした機能を重点的に開発した。
ローカルデータベース開発は、読み出し試験の結果を管理したいという要求から始まり、現在はそれ以外の品質試験も含め、全ての結果や組み立て工程の管理も目標としている。
今後の開発課題として以下の機能をあげる。
\begin{itemize}
  \item 読み出し試験以外(外観検査、平坦性測定等)の結果同期機能.
  \item 中央データベースからローカルデータベースへ品質試験結果の同期.
  \item 品質試験結果解析とモジュール選別機能. 
  \item 組み立て工程管理を世界的にサポート.
  \begin{itemize}
    \item モジュールの組み立て工程は各生産現場ごとに異なる。そのため全ての現場における工程を調査し、それをサポートするシステムとなるように実装する必要がある。
  \end{itemize}
\end{itemize}


