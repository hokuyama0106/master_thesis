\chapter*{Abstract}

The ATLAS Experiment is being conducted with the Large Hadron Colider at CERN. Its purposes are measurement of the Standard Model (SM) and searches for particles beyond the SM.

In order to acquire more statistics and to achieve more advanced measurements and searches, LHC is plannig to increase the luminosity, referred to as HL-LHC.
The target luminosity and integrated luminosity is approximately seven times and ten times higher respectively.

Due to the upgrade, it is required for detectors to have more radiation tolerance and high granularity. 
It is planed to replace the ATLAS inner detector to the new one, referred to as the Inner Tracker(ITk). 
ITk entirely consists of silicon detestors and covers much more wider solid angle acceptance than the current Inner Detector.

For the production of ITk, we are planning to produce $O(10,000)$ modules and conduct a series of tests for quality control(QC tests) for individual modules. 
Those are carried out repeatedly in the production flow.
All the QC tests should be stored to a central database, which is set up at Unicorn university in Czeck Republic, to record the performance of modules itself as the reference for the operation of the ITk. 
I have developed a ``local database'' system in order to manage data at local production sites and to synchronize informations to check the central DB. The system is now under test-use among production sites towards full production.

For spreading the DB system, I organized a tutorial for users at CERN in February 2020. 

Concering the development of the DB system, I have implemented key functions for the production, for example searching results, synchronizing data between the local and the central database, and validated that we can use the whole functionalities of the system, including my developed tools, using devices at the laboratory. 
Additionally I confirmed that we can use the tools with the actual amount of data at the production to measure the processing time of the services.

